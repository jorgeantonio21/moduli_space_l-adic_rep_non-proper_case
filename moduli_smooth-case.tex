\documentclass[10pt,a4paper]{amsart}
\usepackage{fullpage}
%\linespread{1.1}
%\usepackage{amsmath, amscd, amssymb, amsthm, latexsym, url, color, todonotes} %pdflscape}
%\usepackage{graphicx}
\usepackage{mathrsfs}
\usepackage[left=1.0in, right=1.0in, top=1in, bottom=1.3in, includefoot, headheight=13.6pt]{geometry}
\usepackage[colorlinks=true,hyperindex,citecolor=blue,linkcolor=black]{hyperref}
\input{xy}
%\cXyoption{all}
%\usepackage{natbib}
\usepackage{cite}
\usepackage{tikz-cd}
%\usepackage{stix}
\setlength{\marginparwidth}{1in}
\usepackage[capitalize]{cleveref}
%\usepackage[utf8]{inputenc}
\usepackage{eucal,times,amsmath,amsthm,amssymb,mathrsfs,stmaryrd,color,enumerate,accents}
\usepackage{tensor}
\usepackage{xypic}
\usepackage{textcomp}
%\RequirePackage[l2tabu,orthodox]{nag} %detect whether obsolete packages are used
%\documentclass[10pt,a4paper,reqno]{amsart} %reqno places equation numbers on the right
%\linespread{1.1}
\usepackage{mathtools,bm,eucal} % math related
%\usepackage{microtype,fixltx2e,lmodern} % latex technical issues
%\usepackage[utf8]{inputenc} % input encoding
\usepackage[T1]{fontenc} % font encoding
\usepackage{enumerate,comment,braket,xspace,csquotes} % utilities
%\usepackage[all,cmtip]{xy} % because the tikzcd options [shift left], [shift right] do not work on arXiv, we switched some diagrams to xymatrix
%\usepackage[centering,vscale=0.7,hscale=0.8]{geometry}
%\usepackage[hidelinks]{hyperref}

 \numberwithin{equation}{subsection}
\newcommand{\heart}{\ensuremath\heartsuit}
\theoremstyle{plain}
\newtheorem{thm-intro}{Theorem}
\newtheorem{theorem}{Theorem}[section]
\newtheorem*{thm*}{Theorem}
%\newtheorem{claim}[thm]{Claim}
\newtheorem{lemma}[theorem]{Lemma}
\newtheorem{fact}[theorem]{Fact}
\newtheorem{prop}[theorem]{Proposition}
\newtheorem{conjecture}[theorem]{Conjecture}
\newtheorem{coro}[theorem]{Corollary}
\newtheorem{assumption}[theorem]{Assumption}
\newtheorem{claim}[theorem]{Claim}
\theoremstyle{definition}
\newtheorem{defi}[theorem]{Definition}
\newtheorem{notation}[theorem]{Notation}
\newtheorem{exem}[theorem]{Example}
\newtheorem{variant}[theorem]{Variant}
\newtheorem{warning}[theorem]{Warning}
\newtheorem{rema}[theorem]{Remark}
\newtheorem{construction}[theorem]{Construction}
\theoremstyle{remark}
\numberwithin{equation}{section}
\newtheorem{question}[theorem]{Question}


\newcommand{\abs}[1]{\vert#1\vert}


%groups
\newcommand{\GL}{\mathrm{GL}}
\newcommand{\GLn}{\mathrm{GL}_n}
\newcommand{\Gk}{G_K}
\newcommand{\Gal}{\mathrm{Gal}}
\newcommand{\anGLn}{\mathbf{GL}_n^\an}

%stacks
\newcommand{\fib}{\mathrm{fib}}
\newcommand{\dLocSys}{\mathbf R \mathbf{LocSys}_{\ell, n}}
\newcommand{\abdLocSys}{\mathbf R \mathbf{LocSys}_{\ell, n, \Gamma}}
\newcommand{\LocSys}{\mathbf{LocSys}_{\ell, n}}
\newcommand{\abLocSys}{\mathbf{LocSys}_{\ell, n, \Gamma}}
\newcommand{\LocSysfr}{\LocSys^{\mathrm{framed}}}
\newcommand{\abLocSysfr}{\abLocSys^{\mathrm{framed}}}

%symbols
\newcommand{\sm}{\mathrm{sm}}
\newcommand{\tr}{\mathrm{tr}}
%\newcommand{\dim}{\mathrm{dim}}


\newcommand{\Map}{\mathrm{Map}}
\newcommand{\Hom}{\mathrm{Hom}}
\newcommand{\ad}{\mathrm{ad}}
\newcommand{\Ex}{\mathrm{Ex}}
\newcommand{\fCoh}{\mathfrak{Coh}^+}

\newcommand{\sX}{\mathscr{X}}
\newcommand{\sY}{\mathscr{Y}}
\newcommand{\sZ}{\mathscr{Z}}

\newcommand{\sfX}{\mathsf{X}}
\newcommand{\sfY}{\mathsf{Y}}
\newcommand{\sfZ}{\mathsf{Z}}
\newcommand{\sfW}{\mathsf{W}}

\newcommand{\Ok}{k^\circ}


\DeclareMathOperator{\Spec}{Spec}
\DeclareMathOperator{\Spf}{Spf}
\DeclareMathOperator{\Sp}{Sp}
%\newcommand{\Spf}{\mathrm{Spf}}
%\newcommand{\Spec}{\mathrm{Spec}}
\newcommand{\st}{\mathrm{st}}
\newcommand{\Der}{\mathrm{Der}}
\newcommand{\trun}{\mathrm{t}}
\newcommand{\spe}{\mathrm{sp}}

\newcommand{\bfA}{\mathfrak{A}_{\Ok}}
\newcommand{\anA}{\mathbf{A}_k}
\newcommand{\anB}{\mathbf{B}_k}
\newcommand{\banA}{\mathbf{A}_{\Ok}}

\newcommand{\cEnd}{\cE \mathrm{nd}}
\newcommand{\End}{\mathrm{End}}
\newcommand{\Mat}{\mathrm{Mat}}
\newcommand{\unr}{\mathrm{unr}}
\newcommand{\Frac}{\mathrm{Frac}}


\newcommand{\tamepi}{\pi_1^{\mathrm{t}}}
\newcommand{\wildpi}{\pi_1^w}
\newcommand{\tame}{\mathrm{tame}}
\newcommand{\Mon}{\mathrm{Mon}}
\newcommand{\grp}{\mathrm{grp}}
\newcommand{\dSt}{\mathrm{dSt}}

\newcommand{\fc}{\mathrm{fc}}
\newcommand{\Sh}{\mathrm{Sh}}
\newcommand{\Ad}{\mathrm{Ad}}

\newcommand{\Def}{\mathbf{\mathrm{Def}}}
\newcommand{\art}{\mathrm{art}}
\newcommand{\cont}{\mathrm{cont}}
\newcommand{\Q}{\mathbb{Q}}
\newcommand{\Ql}{\bQ_\ell}

\newcommand{\aut}{\mathfrak{aut}}
%homotopy types


%pregeometries and local structures and algebraic categories

\newcommand{\Tdisc}{\mathcal{T}_{\mathrm{disc}}}
\newcommand{\Tet}{\mathcal{T}_{\text{\'et}}}
\newcommand{\Tetph}{\mathcal{T}_{\emph{\text{\'et}}}}
\newcommand{\Tzar}{\mathcal{T}_{\mathrm{Zar}}}
\newcommand{\Tan}{\mathcal{T}_{\an}(k)}
\newcommand{\Tad}{\mathcal{T}_{\ad}(\Ok)}

\newcommand{\smCAlg}{\mathcal{C}\mathrm{Alg}^\sm}
\newcommand{\CAlg}{\mathcal{C}\mathrm{Alg}}
\newcommand{\adCAlg}{\cC \mathrm{Alg}^{\ad}_{\Ok}}
\newcommand{\admCAlg}{\cC \mathrm{Alg}^{\mathrm{adm}}_{\Ok}}
\newcommand{\fCAlg}{\mathrm{f} \cC \mathrm{Alg}_{\Ok}}
\newcommand{\ftaftCAlg}{\fCAlg^{\taft}}
\newcommand{\taft}{\mathrm{taft}}
\newcommand{\AnRing}{\mathrm{AnRing}_k}
\newcommand{\wCAlg}{\cC \mathrm{Alg}^{\wedge}_{\Ok}}



\newcommand{\Str}{\mathrm{Str}}
\newcommand{\locStr}{\mathrm{Str}^{\mathrm{loc}}}
\newcommand{\adm}{\mathrm{adm}}




\newcommand{\Cat}{\mathcal{C}\mathrm{at}_\infty}
\newcommand{\Coh}{\mathrm{Coh}^+}
\newcommand{\Coheart}{\mathrm{Coh}^{+, \heartsuit}}
\newcommand{\bCoh}{\mathrm{Coh}^\mathrm{b}}
\newcommand{\cHom}{\mathcal{H} \mathrm{om}}
\newcommand{\loc}{\mathrm{loc}}



%infty categories, spaces, modules
\newcommand{\op}{\mathrm{op}}
\newcommand{\der}{\mathrm{der}}
\newcommand{\fSpAb}{\mathrm{Sp} \left( \mathrm{Ab}( \fstrfs) \right) }
\newcommand{\Psh}{\mathrm{PShv}}
\newcommand{\Shv}{\mathrm{Shv}}
\newcommand{\ind}{\mathrm{Ind}}
\newcommand{\pro}{\mathrm{Pro}}
\newcommand{\Perf}{\mathrm{Perf}}
\newcommand{\infcat}{$\infty$-category\xspace}
\newcommand{\infcats}{$\infty$-categories\xspace}
\newcommand{\sMap}{\underline{\map}}




\newcommand{\PerfSys}{\mathbf{PerfSys}_{\ell}}
\newcommand{\rigCat}{\mathcal{C}\mathrm{at}^{\mathrm{st}, \omega, \otimes}_{\infty}}



%cohomology
\newcommand{\coho}{C^*_{\mathrm{cont}}( \K, \mathrm{Ad}(\rho))}\newcommand{\cohon}{C^*_{\mathrm{cont}}( \K, \mathrm{Ad}(\rho_n))}
\newcommand{\proet}{\text{pro\'et}}
\newcommand{\et}{\text{\'et}}
\newcommand{\emphet}{\emph{\text{\'et}}}
\newcommand{\Sym}{\mathrm{Sym}}
\newcommand{\dR}{\mathrm{dR}}

%functors
\newcommand{\tcomp}{(-)^\wedge_t}
\newcommand{\functor}{(-)}
\newcommand{\rigg}{(-)^{\mathrm{rig}}}
\newcommand{\rig}{\mathrm{rig}}
\newcommand{\alg}{\mathrm{alg}}
\newcommand{\Fun}{\mathrm{Fun}}
\newcommand{\Hub}{(-)^+}
\newcommand{\sh}{\mathrm{sh}}
\newcommand{\disc}{\functor^\mathrm{disc}}


%for categories of geometric objects
\newcommand{\Top}{\tensor[^{\mathrm{R}}]{\cT \op}{}}
\newcommand{\TopT}{\tensor[^{\mathrm{R}}]{\cT \op}{}( \Tau)}
\newcommand{\adTop}{\tensor[^{\mathrm{R}}]{\cT \op}{}( \Tad)}
\newcommand{\anTop}{\tensor[^{\mathrm{R}}]{\cT \op}{}( \Tan)}
\newcommand{\etTop}{\tensor[^{\mathrm{R}}]{\cT \op}{} (\Tet(\Ok))}
\newcommand{\etphTop}{\tensor[^{\mathrm{R}}]{\cT \op}{} (\Tetph(\Ok))}
\newcommand{\etTopk}{\tensor[^{\mathrm{R}}]{\cT \op}{}(\Tet(k))}
\newcommand{\discTop}{\tensor[^{\mathrm{R}}]{\cT \op}{}(\Tdisc(\Ok))}
\newcommand{\discTopk}{\tensor[^{\mathrm{R}}]{\cT \op}{}(\Tdisc(k))}
\newcommand{\discTopn}{\tensor[^{\mathrm{R}}]{\cT \op}{} (\Tdisc(\Ok_n))}

\newcommand{\Mod}{\mathrm{Mod}}
\newcommand{\St}{\mathrm{St}}

\newcommand{\Afd}{\mathrm{Afd}}
\newcommand{\Afdl}{\mathrm{Afd}_{\Q_\ell}}
\newcommand{\An}{\mathrm{An}}
\newcommand{\Anl}{\mathrm{An}_{\Q_\ell}}
\newcommand{\dAnl}{\mathrm{dAn}_{\Q_\ell}}
\newcommand{\dAfd}{\mathrm{dAfd}}
\newcommand{\dAfdl}{\mathrm{dAfd}_{\Q_\ell}}
\newcommand{\dAn}{\mathrm{dAn}}
\newcommand{\Aff}{\mathrm{Aff}_k}
\newcommand{\dAff}{\mathrm{dAff}_k}
\newcommand{\dSch}{\mathrm{dSch}_k}
\newcommand{\Sch}{\mathrm{Sch}_k}
\newcommand{\dfSch}{\mathrm{dfSch}}
\newcommand{\fSch}{\mathrm{fSch}_{\Ok}}
\newcommand{\dfAff}{\mathrm{dfAff}_{\Ok}}
\newcommand{\dfDM}{\mathrm{dfDM}}
\newcommand{\fDM}{\mathrm{fDM}_{\Ok}}
\newcommand{\an}{\mathrm{an}}
%\newcommand{\Sp}{\mathrm{Sp}}
\newcommand{\Ab}{\mathrm{Ab}}
%\newcommand{\Set}{\mathrm{Set}}

%rings
\newcommand{\Zl}{\mathbb{Z}_{\ell}}


\DeclareMathOperator*{\colim}{colim}

% \mathrm
\newcommand{\rmA}{\mathrm{A}}
\newcommand{\rmB}{\mathrm{B}}
\newcommand{\rmC}{\mathrm C}
\newcommand{\rmD}{\mathrm D}
\newcommand{\rmE}{\mathrm E}
\newcommand{\rmF}{\mathrm F}
\newcommand{\rmG}{\mathrm G}
\newcommand{\rmH}{\mathrm H}
\newcommand{\rmI}{\mahtrm I}
\newcommand{\rmJ}{\mathrm J}
\newcommand{\rmL}{\mathrm{L}}
\newcommand{\rmP}{\mathrm P}
\newcommand{\rmR}{\mathrm R}
\newcommand{\rmT}{\mathrm T}

% \mathbb
\newcommand{\bA}{\mathbb A}
\newcommand{\bB}{\mathbb B}
\newcommand{\bC}{\mathbb C}
\newcommand{\bD}{\mathbb D}
\newcommand{\bE}{\mathbb E}
\newcommand{\bF}{\mathbb F}
\newcommand{\bL}{\mathbb L}
\newcommand{\bQ}{\mathbb Q}
\newcommand{\bR}{\mathbb{R}}
\newcommand{\bT}{\mathbb T}
\newcommand{\bZ}{\mathbb Z}

%\mathfrak
\newcommand{\fA}{\mathfrak A}
\newcommand{\fB}{\mathfrak B}
\newcommand{\fC}{\mathfrak C}
\newcommand{\ff}{\mathfrak f}
\newcommand{\fX}{\mathfrak X}
\newcommand{\fY}{\mathfrak Y}
\newcommand{\fZ}{\mathfrak Z}
\newcommand{\fg}{\mathfrak g}
\newcommand{\fm}{\mathfrak{m}}
\newcommand{\fl}{\mathfrak l}

%\mathcal
\newcommand{\cA}{\mathcal A}
\newcommand{\cB}{\mathcal B}
\newcommand{\cC}{\mathcal C}
%\newcommand{\cD}{\mathcal D}
\newcommand{\cE}{\mathcal E}
\newcommand{\cF}{\mathcal{F}}
\newcommand{\cG}{\mathcal{G}}
\newcommand{\cI}{\mathcal{I}}
%\newcommand{\cH}{\mathcal{H}}
\newcommand{\cK}{\mathcal{K}}
%\newcommand{\cL}{\mathcal{L}}
\newcommand{\cM}{\mathcal M}
\newcommand{\cN}{\mathcal N}
\newcommand{\cO}{\mathcal{O}}
\newcommand{\cP}{\mathcal{P}}
\newcommand{\cQ}{\mathcal{Q}}
\newcommand{\cT}{\mathcal{T}}
\newcommand{\cX}{\mathcal X}
\newcommand{\cY}{\mathcal Y}
\newcommand{\cZ}{\mathcal Z}
\newcommand{\cS}{\mathcal S}

%\widehat
\newcommand{\hA}{\widehat{A}}
\newcommand{\hB}{\widehat{B}}
\newcommand{\hAA}{\widehat{A'}}
\newcommand{\hAp}{\widehat{A}_X[p^{-1}]}
\newcommand{\hbZ}{\widehat{\bZ}}
%\newcommand{\abs}[1]{\vert#1\vert}
\newcommand{\overK}{\overline{K}}












\usepackage[english]{babel}




\author{Jorge Ant\'onio}
%\thanks{L'auteur a b�n�fici� du soutient du projet ANR-10-BLAN-0114 "ArShiFo"}
\address{Jorge Ant\'onio,  IMT Toulouse, 118 Rue de Narbonne  31400 Toulouse}
\email{jtiago1993@gmail.com}
\thanks{To all}

\begin{document}

\title{Moduli of $\ell$-adic representations (Continuation)}




\date{\today}

\maketitle

\renewcommand\labelitemi{\textbullet}






\markright{MODULI OF $\ell$-ADIC REPRESENTATIONS}


\begin{abstract}
In this text we prove that if we take $G$ is a more general profinite group, for example an absolute Galois group, $G$, the moduli $\mathrm{LocSys}_{G,n}^{ \Gamma}$ is representable by a rigid $\ell$-analytic space, provided we fix the inertia action at infinity.
\end{abstract}

\setcounter{tocdepth}{1}
\tableofcontents


\section*{Introduction}
In this short text we extend some of the results of \cite{me1}. Namely, our goal is to construct the (derived) moduli stack of $\ell$-adic pro-\'etale local systems on a smooth variety $X$, $\LocSys(X)$ over an algebraically closed field of positive characteristic.
The non-proper case requires special care as the \'etale fundamental group $\pi_1^\et(X)$ is not, in general, topological of finite type. Thus the results in \cite{me1} do not apply in this context and the moduli $\LocSys(X)$ is generally not representable
by a $\Q_\ell$-analytic stack.

Nonetheless, let $\Gamma$ be a finite quotient of the wild fundamental group of $X$, $\pi_1^w(X)$. We can consider the substack $\abLocSys(X)  $ of $\LocSys(X)$ parametrizing $\ell$-adic pro-\'etale local systems on $X$ with ramification
bounded by $\Gamma$, at infinity. We prove the following:

\begin{prop}

\end{prop}


In the non proper case, the \'etale fundamental group $\pi_1^\et(X)$ is not necessarily topologically of finite generation, therefore one cannot hope that $\dLocSys$ is representable by a derived $\Q_\ell$-analytic stack. Even though $\dLocSys$ is not
representable one can prove that it admits sub-stacks $\abdLocSys \subseteq \dLocSys$ which they themselves are representable by derived $\Q_\ell$-analytic stacks: let $\pi_1^w (X) \subseteq \pi_1(X)$ denote the \emph{wild fundamental group} of $X$.
In general, it is huge a pro-$p$ group but given a finite quotient $p \colon \pi_1^w(X) \to \Gamma$ one can consider the derived moduli stack $\abdLocSys$ parametrizing continuous representations
	\[
		\rho \colon \pi_1^\et(X) \to \GLn(\overline{\Q}_\ell),
	\]
whose restriction to $\pi_1^w(X)$ factor through the quotient $p \colon \pi_1 ^w(X) \to \Gamma$. More explicitly, $\abdLocSys$ parametrizes those $\rho$ for which one has a framing 
	\begin{equation} \label{frame}
		\alpha_{\rho, \Gamma} \colon \rho_{|_H} \simeq \mathrm{triv}
	\end{equation}
where $H \unlhd \pi_1^w(X)$ denotes the kernel of $p$ and $\mathrm{triv}$ denotes the trivial representation $\pi_1^\et(X) \to \GLn(\overline{\Q}_\ell)$. Let us give a brief review of the contents of each section of the text.
In section 2.1, we recall some of the main aspects of the ramification theory in both the local case, i.e. for absolute Galois groups of local fields. Our exposition is classical and in no way we prove anything
new in this context. Section 2.2 is devoted to a brief exposition on the ramification theory for non-proper smooth schemes over algebraically closed fields. 
In section 2.3
we construct the (ordinary) \emph{moduli stack of continuous $\ell$-adic representations}. Our construction follows directly the methods applied in \cite{me1}. Given $p \colon \pi_1^w(X) \to \Gamma$ a continuous group
homomorphism whose target is finite we construct the moduli stack $\abLocSys$ of $\ell$-adic continuous representations of $\pi_1^et(X)$ equipped with a frame as in \eqref{frame} and we finally show that in such case $\abLocSys$ is representable
by a $\Q_\ell$-analytic stack (the analogue of an Artin stack in the context of rigid analytic geometry), i.e., we prove the following result:

\begin{theorem}[\cref{main1}]
The stack $\abLocSys(X)$ admits a smooth atlas by a $\Q_\ell$-analytic space 
	\[
		\pi \colon \cX \to \abLocSys(X)
	\]
and therefore is representable by a $\Q_{\ell}$-analytic stack. We can moreover identify the functor of points of $\cX$: given $A \in \Afd^{\op}$ a
$\Q_\ell$-affinoid algebra, we have a natural bijection of sets
	\begin{align}
		\Hom_{\mathrm{An}} \left( \Sp (A), \cX \right) & \cong \\
		& \cong \Hom_{\cont} \left( \pi_1^\emph{\et}(X), \GLn(A) \right) \times_{ 		\Hom_{\cont} \left( \pi_1^w(X), \GLn(A) \right) 		} 		\Hom_{\cont} \left( \Gamma, \GLn(A) \right).
	\end{align}
\end{theorem}

In section 3, we upgrade canonically both the stacks $\LocSys(X)$ and $\abLocSys(X)$ to derived moduli stacks which we denote by $\dLocSys(X)$ and $\abdLocSys(X)$, respectively. We are able to compute their corresponding tangent complexes:

\begin{prop}
Let $\rho \in \dLocSys(X)(\overline{\Q}_\ell)$, then we have a canonical morphism
	\[
		\theta_\rho \colon \mathbb T_{\dLocSys(X), \rho} \simeq C^*_{\emph{\et}} \left(X, \Ad(\rho) \right) [1]
	\]
which is an equivalence in the \infcat $\Mod_{\overline{\Q}_\ell}$. Moreover, if $\rho \in \abdLocSys(X)(\overline{Q}_\ell)$ then the natural morphism 
	\[
		\mathbb T_{\abdLocSys(X), \rho} \to \mathbb T_{\dLocSys(X), \rho}
	\]
is an equivalence in the \infcat $\Mod_{\Q_\ell}$.
\end{prop}


\begin{theorem} \label{thm:der}
The derived moduli stack $\abdLocSys(X) \in \dSt \left( \dAfdl, \tau_{\et}, P_{\sm} \right)$ is representable by a \emph{derived $\Q_{\ell}$-analytic stack}.
\end{theorem}

The derived structure on $\dLocSys$ allow us to prove the following result:

\begin{prop}[\cref{open_im}] \label{op}
There is a canonical inclusion morphism of derived moduli stacks
	\[
		\iota \colon \coprod_{\overline{\rho}} \Def^{\rig}_{\overline{\rho}} \to \dLocSys(X),
	\]
where the coproduct runs over all residual continuous representations $\overline{\rho} \colon \pi_1^{\emph{\et}}(X) \to \GLn(	\overline{\mathbb{F}}_\ell	)$ and $\Def^{\rig}_{\overline{\rho}}$ denotes the derived moduli stack parametrizing continuous
representations $\rho \colon \pi_1^{\emph{\et}} (X) \to \GLn(\Q_\ell)$ which are correspond to suitable $\ell$-adic deformations of $\overline{\rho}$. Moreover, $\iota$ is an admissible immersion, i.e.,, for each such $\overline{\rho}$ the derived moduli stack
$\Def^{\rig}_{\overline{\rho}}$ is an admissible derived substack of $\LocSys(X)$.
\end{prop}

\cref{op} gives us a comparison of our construction with already existing construction such as those in \cite{chenevier}, \cite{mazurDG} and \cite{galatius_dg}. The morphism $\iota$ is an epimorphism of stacks when we consider rank $n=1$ $\ell$-adic continuous
representations. However, it $\iota$ is not epi in general as the following example illustrates:

\begin{exem}
Let $G= \bZ_\ell$ and consider the following rank $2$ $\ell$-adic continuous representation of $\rho \in \dLocSys(\bZ_\ell) \left(\Q_\ell \langle T \rangle \right)$:
	\[
		\rho \colon \bZ_\ell \to \GL_2 \left( \Q_\ell \langle T \rangle \right)
	\]
given by the association
	\[
		1 \Mapsto 
		\begin{bmatrix}
			1 & T \\
			0 & 1
		\end{bmatrix}.
	\]
Then $\rho$ does not belong to the essential image of the morphism $\iota \colon \coprod_{\overline{\rho}} \Def^{\rig}_{\overline{\rho}} \to \dLocSys$.
\end{exem}

\begin{rema}
The morphism $\iota$ is not far from being an equivalence of derived moduli stacks. Therefore, the geometry of $\dLocSys(X)
$ is \emph{too discrete} for many purposes, in particular $\dLocSys$ is not almost of finite presentation. For example, the de Rham cohomology of $\dLocSys$, defined in terms of the derived de Rham
complex, is bad behaved. It would be interesting to be able to find a morphism $\varphi \colon \dLocSys(X) \to \cX$, where $\cX$ is a derived $\Q_\ell$-analytic stack such that $\cX$ is almost of finite presentation and is constructed from $\dLocSys$ by a
gluing the formal neighborhoods $\Def^\rig_{\overline{\rho}}$ together. This is probably too much to hope for, but some results have been done in this direction in the local case for $p = \ell$, see \cite{gee}.
\end{rema}

This is the content of section 4.
 Lastly, section 5 deals with the existence of shifted symplectic forms on
the derived moduli stack $\dLocSys(X)$. We will use the existence of shifted symplectic structures on derived moduli stacks underlying $\rigCat$-valued derived stacks on $(\dAfdl, \tau_{\et})$, see \cref{rigCat}, proven in \cite{toen_ss}. Our main result is the
following:

 \begin{theorem} \label{int_thm1}
 The derived moduli stack $\dLocSys (X)$ admits a canonical $2-2d$-shifted symplectic structure, induced by Poincar\'e duality.
 \end{theorem}

\subsection{Convention and Notations} We denote $\Afd$ the category of affinoid $\Q_\ell$-spaces, $\dAfdl$ the \infcat of derived $\Q_\ell$-affinoid spaces, $\An$ the category of analytic $\Q_\ell$-spaces and $\dAnl$ the \infcat of derived $\Q_\ell$-analytic
spaces, see \cite[Definition 7.3]{porta_der}. We will denote $\cI \cP(\S) \coloneqq \ind \left( \pro \left( \S \right) \right)$ the \infcat of ind-pro objects in the \infcat of spaces $\S$. We denote $ \cE \Cat (\cI \cP(\S))$ the \infcat of
small $\cI \cP (\S)$-enriched \infcats.
where $\mathrm{Ad} \left( \rho \right) \simeq \rho \otimes \rho^\vee$ denotes the adjoint representation, $A_Z$ denotes the underlying (derived) algebra associated to $Z$ and $C^*_{\et}\left(X, \mathrm{Ad} \left( \rho \right) \right) \in \mathrm{Mod}_{A_Z}$ denotes
the chain complex of (pro-)\'etale cohomology of $X$ with $\mathrm{Ad} (\rho)$-coefficients. 


\section{Previous works and Rappels}
Let $X$ be a smooth and proper scheme over an algebraically closed field of characteristic $p>0$. In \cite{me1} it was constructed the moduli of continuous $\ell$-adic representations, for $\ell$ not necessarily different than $p$, of $\pi_1^{\et}(X)$.
One has proven the following result:

\begin{theorem} \label{before}
The moduli stack $\LocSys(X) \colon \Afd \to \S$ is representable by a $\Q_\ell$-analytic stack.
\end{theorem}

In this case \cref{before} holds mainly because the profinite group $\pi_1^{\et}(X)$ is topologically of finite type, i.e., it can be generated by a finite number of generators seen as a profinite group. 
\section{Setting the stage}

\subsection{Recall on the monodromy of (local) inertia} In this subsection we recall some well known facts on the monodromy of the local inertia, our exposition relies heavily on \cite{fontaine_ouyang}.


Let $K$ be a local field, $\O_K$ its ring of integers and $k$ its residue field which we assume to be of characteristic $p>0$ different from $\ell$. Fix $\overline{K}$ an algebraic closure of $K$ and denote by $\Gk \coloneqq
 \Gal \left( \overK / K \right)$ its absolute Galois group.
 
 \begin{defi}
Given a finite Galois extension $L/K$ with Galois group $\Gal \left( L / K \right)$ we define its \emph{inertia group}, denoted $I_{L/K} $, as the subgroup of $\Gal \left( L / K \right)$
spanned by those elements of $\Gal \left( L /K \right)$ which act trivially on $\mathfrak l \coloneqq \cO_L / \fm_L$, where $\O_L $ denotes the ring of integers of $L$ and $ \mathfrak{m}_L$ the corresponding maximal ideal. 
\end{defi}

\begin{rema}
It follows from the definitions we can identify the inertia subgroup $ I_{L/K}$ of $\Gal( L / K )$ as the kernel of the surjective group homomorphism $q \colon \Gal( L / K ) \to \Gal( l/ k)$. We have thus a short exact sequence
	\begin{equation} \label{inertia}
		1 \to I_{L/K} \to \Gal( L / K ) \to \Gal( l/ k) \to 1,
	\end{equation}
of profinite groups. In particular,
as $I_{L / K }$ can be identified with the kernel of the morphism $q$ it follows that it is a normal subgroup of $\Gal( L / K)$.
\end{rema}

\begin{rema}
Varying the finite extension $L/K$ we can assemble
together the short exact sequences displayed in \eqref{inertia} to obtain a short exact sequence
	\begin{equation} \label{absinert}
		1 \to I_K \to \Gk \to G_k \to 1
	\end{equation}
where $G_k \coloneqq \Gal( \overline{k}/ k)$ for a fixed algebraic closure $\overline{k}$ of $k$.
\end{rema}

\begin{defi}[Absolute inertia]
Define the \emph{(absolute) inertia group of $K$} as the inverse limit
	\[
		I_K := \lim_{L/K \text{ finite}} I_{L/ K},
	\]
which we can canonically identify with a subgroup of $\Gk$.
\end{defi}

It turns out that in nature inertia occurs in the wild:
given $L/ K$ as above we can consider the subgroup $P_{L/K}$ of $I_{L/K}$ spanned by those elements which act trivially on $\cO_L/ \fm^2_L$, which we designate the \emph{wild inertia group} associated to the algebraic extension $L/K$ (or simply the wild inertia
of $L/
K$). 

\begin{defi}[Absolute wild inertia]
We define the absolute wild inertia group of $K$ as:
	\[
		P_K : = \lim_{L \text{ finite}} P_{L/K}.
	\] 
\end{defi}


\begin{rema}
It follows from the definitions that $P_K $ can be identified with a normal subgroup of $I_K$.
\end{rema}

Consider the exact sequence
	\begin{equation} \label{wild_tame}
		1 \to P_K \to I_K \to I_K / P_K \to 1.
	\end{equation}
Thanks to \cite[Lemma 53.13.6]{stacks} it follows that the wild inertia group $P_K$ is a \emph{pro-$p$} group and it is not topologically finitely generated in general. However, the quotient $I_K / P_K$ is much more amenable:

\begin{prop}{\cite[Corollary 13]{bommel}} \label{tame_mon} Let $p \coloneqq \mathrm{char}(k)$ denote the residual characteristic of $K$.
The quotient $I_K / P_K$ is canonically isomorphic to $\hbZ'(1)$, where the latter denotes the profinite group $\prod_{q \neq p} \bZ_q(1)$. In particular, the quotient profinite group $I_K / P_K$ is topological of finite generation.
\end{prop}


Define $P_{K, \ell} $ to be the inverse image of $\prod_{q \neq \ell, p } \bZ_q$ in $I_K$. For this reason, we have a short exact sequence of (profinite) groups
	\[
		1 \to P_K \to P_{K, \ell} \to \prod_{q \neq \ell, p } \bZ_q \to 1.
	\]
Define similarly $G_{K, \ell} \coloneqq G_K / P_{K, \ell}$ the quotient of $G_K$ by $P_{K, \ell}$. We have a short exact sequence of (profinite) groups
	\begin{equation} \label{e1}
		1 \to P_{K, \ell} \to \Gk \to G_{K, \ell} \to 1,
	\end{equation}
and putting together \eqref{wild_tame} with \cref{tame_mon} we obtain a short exact sequence
	\begin{equation} \label{e2}
		1 \to \bZ_\ell(1) \to G_{K, \ell} \to G_k \to 1.
	\end{equation}
	
\begin{rema}
As a consequence of both \eqref{e1} and \eqref{e2} the quotient $G_{K, \ell}$ is topologically of finite type.
\end{rema}

Suppose we are now given a continuous representation
	\[
		\rho \colon \Gk \to \GLn ( \mathbb{Q}_\ell),
	\]
(we can also consider $\rho$ with values in a finite extension $E$ of $\mathbb{Q}_\ell$, without changing the exposition). Up to conjugation we can suppose that $\rho$ preserves a lattice inside the vector space underlying $\rho$, thus we have a
commutative diagram

	\[
	\begin{tikzcd}
		G_K \ar{r}{\widetilde{\rho}} \ar{dr}[swap]{\rho} & \GLn(\bZ_\ell) \ar{d} \\
						& \GLn(\Q_\ell)
	\end{tikzcd}.
	\]
	
For this reason, $\widetilde{\rho} \left( G_K \right) $ is a closed subgroup of $\GLn(\bZ_\ell)$. We have moreover, a short exact sequence

	\[
		1 \to N_1 \to \GLn(\bZ_\ell) \to \GLn(\bF_\ell) \to 1,
	\]
where $N_1$ is a pro-$\ell$-subgroup of $\GLn(\bZ_\ell)$, more explicitly it consists of the subgroup of $\GLn(\bZ_p)$ formed by those matrices congruent to $\mathrm{Id}$ mod $\ell$.
By construction $P_{K, \ell}$ is a profinite group for which every finite quotient if of order prime to $\ell$. Thus one has necessarily
	\[
		\rho \left( P_{K, \ell}  \right) \cap N_1 = \{1 \}.
	\] 
We conclude that $\rho( P_{K, \ell})$ injects into the finite group $\GLn(\bF_\ell)$. Which in turn implies that the (absolute) wild inertia group $P_K$ itself acts on $\GLn(\Q_\ell)$ via a finite quotient. 

\subsection{Geometric \'etale fundamental groups}
Let $X$ be a geometrically connected smooth scheme over an algebraically closed field $k$ which we fix throughout this section except otherwise stated.
Fix once and for all a geometric point $\iota_x \colon \overline{x} \to X$ and consider the corresponding \'etale fundamental group $\pi_1^{\et}(X) \coloneqq \pi_1^{\et}(X, \overline{x})$, which is a profinite group. If we assume $X$ proper one has the following
classical result:

\begin{theorem}{\cite[Expos\'e 10, Thm 2.9]{grothendieckSGA1}} \label{proper_case}
The \'etale fundamental group $\pi^\emph{\et}_1 \left( \overline{X} \right)$ is topologically of finite type.
\end{theorem}

Unfortunately, the statement of \cref{proper_case} does not hold in the non-proper case as the following proposition illustrates:

\begin{prop}
Let $k$ be an algebraically closed field of positive characteristic. Then the \'etale fundamental group of the affine line $\mathbb A^1_k$, denoted $\pi_1^\emph{\et}(\mathbb A^1_k)$, is not topologically finitely generated.
\end{prop}

\begin{proof}
For each integer $n \geq 1$, one can exhibit Galois covers of $\mathbb A^1_k$ whose corresponding automorphism group is isomorphic to $\left( \mathbb Z / p \mathbb Z \right)^n$. This implies immediately that $\pi_1^{\et}(\mathbb A^1_k)$ does not admit a finite
number of
topological generators. In order to construct such coverings, we consider the following endomorphism of the affine line
	\[
		\phi_n \colon \mathbb A^1_k \to \mathbb A^1_k,
	\]
defined via the formula
	\[
		\phi_n \colon x \Mapsto x^{p^n} - x.
	\] 
The endormophism $\phi_n$ respects the additive group structure on $\mathbb A^1_k$. Moreover,
the differential of $\phi_n$ equals $-1$. For this reason, $\phi_n$ induces an isomorphism on cotangent spaces and in particular it is an \'etale morphism.
As $k$ is algebraically closed, $\phi_n$ is surjective and it is finite, thus a finite \'etale covering. The automorphism group
of
$\phi_n$ is naturally
identified with its kernel, which is isomorphic to $\mathbb F_{p^n}$. The statement of the proposition now follows.
\end{proof}

\begin{defi}
Let $G$ be a profinite group and $p$ a prime numbger, we say that $G$ is \emph{quasi-$p$} if $G$ equals the subgroup generated by all $p$-Sylow subgroups of $G$.
\end{defi}

Examples of quasi-$2$ finite groups include the symmetric groups $S_n$, for $n \geq 2$. Moreover, for each prime $p$, the group $\mathrm{SL}_n(\mathbb F_p)$ is quasi-$p$.
Let $X = \mathbb A^1_k$ be the affine line over an algebraically closed field $k$ of characteristic $p>0$. 
We have the following result proved by Raynaud which was originally a conjecture of Abhyankar:

\begin{theorem}{\cite[Conjecture 10]{Clark}} 
Every finite quasi-$p$ group can be realized as a quotient of $\pi_1^{\emph{\et}} \left(X \right)$.
\end{theorem}

In the example of the affine line the infinite nature of $\pi_1(\mathbb A^1_k )$ arises as a phenomenon of the existence of \'etale coverings whose ramification at infinity can be as large as we desire. This phenomenon is special to the positive characteristic
setting.
Neverthless, we can prove that $\pi_1^{\et}(X)$ admits a topologically finitely generated quotient which corresponds to the group of automorphisms of tamely ramified coverings. Needless to say that in the proper case every finite \'etale covering is everywhere
unramified.

\begin{defi}
Let $X \hookrightarrow \overline{X}$ be a normal compactification of $X$, whose existence is guaranteed by \cite{nagata}. Given a finite \'etale cover $f \colon Y \to X$, with $Y$ connected,
we say that $f$ is \emph{tamely ramified along} the divisor $D : = \overline{X} \backslash X$ if every codimension-$1$ point $x \in D$ is
tamely ramified in the resulting extension of \emph{generic} fields $k(Y) / k(X)$. Denote by $\pi_1^w(X, D)$
the kernel of the continuous morphism $ \pi_1^{\et}(X) \to \tamepi(X)$.
\end{defi}

\begin{fact}
Tamely ramified extensions along $D \coloneqq \overline{X} \backslash X$ of $X$ are classified by a quotient $\pi_1^{\emph{\et}}(X) \to \tamepi(X, D)$, we refer to the latter profinite group as the \emph{fundamental tame group along $D$}.
\end{fact}


\begin{defi} \label{}
\begin{enumerate}
\item Let $f \colon Y \to X$ be an \'etale covering. We say that $f$ is \emph{divisor-tame} if for every normal compactification
$X \hookrightarrow \overline{X}$, $f$ is tamely ramified along $D = \overline{X} \backslash X$.
\item The \emph{tame fundamental group} $\tamepi(X)$ is defined as the quotient of $\pi_1^\et(X)$ by the normal closure of
those opens subgroups $\pi_1^w(X, D)$, for each normal compactification $X \hookrightarrow \overline{X}$. It classifies tamely ramified \'etale morphisms over $X$.
\end{enumerate}
\end{defi}

\begin{fact}
When $X$ is a smooth curve over $\overline{k} = k$, then the notion of divisor-tameness coincides with the usual notion of curve tameness, see \cite[Appendix 1]{cadoret}.
\end{fact}

\begin{rema}
The tame fundamental group $\tamepi(X)$ classifies those finite \'etale coverings $f \colon X \to Y$ which are tamely ramified
along any divisor at infinity. Moreover, under the assumption that $X$ is smooth divisor tameness can be characterized
by tameness with respect to morphisms $C \to X$, where $C$ is a smooth curve, see \cite[Appendix 1]{cadoret}.
\end{rema}

\begin{defi}
We define the \emph{wild fundamental group} of $X$, denoted $\pi_1^w(X)$, as the kernel of the surjection $\pi_1^{\et}(X) \to
\tamepi(X)$. It is an open normal subgroup of $\pi_1^{\et}(X)$.
\end{defi}

\begin{fact}
Let $C$ be a geometrically connected smooth curve over $k$. The wild fundamental group $\pi_1^w(X)$ is a pro-$p$-group.
\end{fact}

\begin{theorem}{\cite[Appendix 1, Theorem 1]{cadoret}} \label{cadoret}
Let $X$ be a smooth and geometrically connected scheme over $k$. There exists a smooth, geometrically connected curve
$C/ k$ together with a morphism $f \colon C \to X$ of varieties such that the corresponding morphism at the level of
fundamental groups $\pi_1^{\emph{\et}}(C) \to \pi_1^{\emph{\et}}(X) \to \tamepi(X)$ is surjective and it factors by a well defined morphism
$\pi_1^{\mathrm{t}}(C) \to \tamepi(X)$. In particular, $\tamepi(X)$ is topologically finitely generated.
\end{theorem}

\begin{rema}
Thanks to \cref{cadoret} it follows that $\tamepi (\mathbb A^1_k)$ is topologically finitely generated. Actually, it turns out that $\tamepi(\mathbb A^1_k)$ is trivial.
\end{rema}

\begin{defi}
Define the $\infty$-topos of tamely ramified local systems on $X$ to be 
\end{defi}

\subsection{Absolute Galois groups of number fields} Let $K / \Q$ be a finite field extension. Given a finite set $S = \{ \mathfrak p_1, \dots, \mathfrak p_m \}$ of prime ideals of $K$, we denote $K_S / K$ the largest Galois field extension of $K$ unramified outside
$S$. We denote $G_{K, S} : = \Gal \left( K_S / K \right)$ denote the corresponding absolute group. We have the following important result:

\begin{prop}
Let $p $ be a prime a number not in $S$, then the pro-$p$ completion of every open subgroup $H $ of $G_{K, S}$ is topologically finitely generated.
\end{prop}

\begin{rema}
Let $\rho \colon G_{K, S} \to \GLn(\Q_\ell)$ be a continuous representation, the inverse image of the open subgroup $N_{\ell, n } ( \Q_\ell) \subseteq \GLn(\Q_\ell)$ by $\rho$ is an open subgroup, say $H$ of $G_{K, S}$. As $N_{\ell, n } ( \Q_\ell) \cong \lim_{n \geq 0}
\mathrm{Id} + \ell^n \mathrm{M}_n( \bZ/ \ell^{n + 1})$ and, for each $n \geq 0 $, $\mathrm{Id} + \ell^n \mathrm{M}_n( \bZ/ \ell^{n + 1})$ is an $\ell$-group. Therefore $\ell$-finiteness for $G_{K, S}$ implies that the restriction
	\[
		\widetilde{\rho }_\ell \colon H \to N_{ \ell, n }(\Q_\ell)
	\]
factors through a topologically finitely generated pro-$\ell$ group.
\end{rema}

\subsection{Moduli of continuous $\ell$-adic representations} \label{section 2.3}
Let $X$ denotes either:
\begin{enumerate}
\item A geometrically connected smooth scheme over an algebraically closed field of characteristic $p>0$ with $p \neq \ell$;
\item The spectrum of a mixed characteristic local field, K, whose residual characteristic is different from $\ell$;
\end{enumerate}

Our exposition will treat homogeneously these two cases. Accordingly, we denote $G_X$ to be either:

\begin{enumerate}
\item The \'etale fundamental group, $\pi_1^\et(X)$, of $X$;
\item The absolute Galois group of $K$, $G_K$;
\end{enumerate}

\begin{rema}
Let $A \in \Afd$ be $\Q_{\ell}$-affinoid algebra $A \in \Afd$. It admits a natural topology induced from a choice of a norm on $A$, compatible with the usual $\ell$-adic valuation on $\Q_\ell$. 
Given $\mathbf G$
an analytic $\Q_\ell$-group space we can consider the corresponding group of $A$-points on $\mathbf G$, denoted $\mathbf G(A)$. The group $\mathbf G (A)$ admits a natural topology induced from the topology on $A$.
In the current text we will be interested in studying the moduli parametrizing continuous representations
	\[
		\rho \colon G_X \to \mathbf{ \mathrm{GL}}^{\an}_n(A).
	\]
Our exposition can be adapted to treat the case of continuous representations of $G_X$ with values in $A$-points of analytifications of reductive group schemes.
\end{rema}

\begin{defi}
Denote by
	\[
		\LocSysfr \colon \Afdl \to  \Set,
	\]
the \emph{functor of rank $n$ continuous $\ell$-adic group homomorphisms of $G_X$}. It is defined by the formula
	\begin{equation} \label{e21}
		A \in \Afd^{\op} \Mapsto \Hom_{\mathrm{cont}} \left( G_X, \GLn( A) \right) \in \Set,
	\end{equation}
where the right hand side of \eqref{e21} denotes the set of continuous group homomorphisms $G_K \to \GLn(A)$.
\end{defi}

\begin{prop}{\cite[Corollary 2.2.16]{me1}} \label{me1}
Suppose $G$ is a topologically finitely generated profinite group. Then the functor $\LocSysfr(G_X)$
is representable by a $\Q_{\ell}$-analytic space. 
\end{prop}

\begin{rema}
Unfortunately, $G_X$ is almost never topologically finitely generated in the non-proper case. For this reason, we cannot expect the functor $\LocSysfr(G_X) $ to be representable by an object in the category $\An_{\Q_\ell}$ of $\Q_\ell$-analytic spaces.
Nevertheless,
we can prove an analogue of \cref{me1} if we consider instead certain subfunctors of $\Hom_{\cont} \left( G_X, \GLn( - ) \right)$. More specifically, we will consider functor parametrizing \emph{pro-\'etale local systems} on $X$ with bounded
ramification at infinity.
\end{rema}

\begin{notation}
In accordance to our convention, we let $\mathrm P_X$ denote either:
\begin{enumerate}
\item The wild inertia fundamental group, $\pi_1^w(X)$;
\item The wild inertia subgroup $P_K$ of $G_K$;
\end{enumerate}
\end{notation}

\begin{construction} \label{const1}
Let $q \colon \mathrm P_X \to \Gamma$ be a surjective continuous group homomorphism, whose target is a finite group (equipped with the discrete topology). We define the \emph{functor of continuous group homomorphisms $G_X$ to $\GLn(-)$
whose ramification at infinity
is bounded by $\Gamma$}, as the fiber product:
	\begin{equation} \label{eq_def}
		\abLocSys^{\mathrm{framed}}(G_X): =  \LocSys^{\mathrm{framed}}(G_X)\times_{  		\LocSys^{\mathrm{framed}}(\mathrm P_X)	}	\LocSys^{\mathrm{frame}}(\Gamma),
	\end{equation}
computed in the category $\Fun \left( \Afd^{\op}, \mathrm{Set} \right)$.
\end{construction}

\begin{rema}
Note that in \cref{const1} our notations depend on the choice of the continuous surjective homomorphism $q \colon \mathrm P_X \to \Gamma$. However, for notational convenience we drop the subscript $q$.
\end{rema}

\begin{theorem} \label{hom_loc}
The functor $\abLocSysfr(G_X)$ is representable by a $\Q_{\ell}$-analytic stack.
\end{theorem}

\begin{proof} Let $r$ be a positive integer and denote $\mathrm F^{[r]}$ a free profinite group on $r$ topological generators.
The finite group $\Gamma$ and the quotient $G_X / \mathrm P_X $ are topologically of finite generation. Therefore, it is possible to choose
a continuous group homomorphism 
	\[
		p \colon \mathrm F^{[r]} \to G_X,
	\]
such that the images $p(e_i)$, for $i = 1, \dots, r$, form a set of generators for $\Gamma$, seen as a quotient of $P_{X}$, and for  $G_X / \mathrm P_X$.
Restriction under $\varphi$ induces a closed immersion of functors 
	\[
		\abLocSysfr(G_X) \hookrightarrow \LocSysfr(\mathrm F^{[r]}).
	\]
Thanks to \cite[Theorem
2.2.15.]{me1} the latter
is representable by a rigid $\Q_\ell$-analytic space, denoted $X^{[r]}$. It follows that $\abLocSysfr(G_X)$ is representable by a closed subspace of $X^{[r]}$, as desired.
\end{proof}

\begin{defi} \label{const_1}
Let $\Psh \left( \Afd
\right)$ denote the \infcat of $\S$-valued preasheaves on $\Afd$.
Consider the Grothendieck site $( \Afd, \tau_{\et})$, where $\tau_{\et}$ denotes the \'etale topology on $\Afd$. We define the \infcat of \emph{higher (\'etale) stacks} on $(\Afd, \tau_{\et})$, denoted $\St \left( \Afd, \tau_{\et} \right),
$ as the full subcategory of $\Psh \left( \Afd \right)$ spanned by those pre-sheaves which satisfy \'etale hyper-descent.
\end{defi}

\begin{rema}
The inclusion functor $ \St \left( \Afd, \tau_{\et} \right) \subseteq \Psh \left( \Afd \right)$ admits a left adjoint, which is a left localization functor. For this reason, the \infcat $\St \left( \Afd, \tau_{\et} \right)$ is a presentable \infcat.
\end{rema}

\begin{defi}
Consider the geometric context $(\dAfd, \tau_{\et}, \mathrm P_{\sm} )$, \cite[Definition 2.3.1]{me1}. Let $\St \left( \Afd, \tau_{\et}, \mathrm P_\sm \right)$ denote the full subcategory of $\St( \Afd, \tau_{\et})$
spanned by geometric stacks, \cite[Definition 2.3.2]{me1}. We will refer to $\St \left( \Afd, \tau_{\et}, \mathrm P_\sm \right)$ as the \infcat of \emph{$\Q_\ell$-analytic stacks}.
\end{defi}

\begin{exem}
Let $\mathbf G$ be a group object in the $\infty$-category $\St \left( \Afd, \tau_{\et}, \mathrm P_{\sm} \right)$. Given a $\mathbf G$-equivariant object $\cF \in \St \left( \Afd, \tau_{\et}, P_{\sm} \right)^{\mathbf G}$ we let $[\cF / \mathbf G ]$ denote the geometric
realization of the simplicial object
	\[
	\xymatrix{
 		\cdots \ar[r]<1.5pt>\ar[r]<-1.5pt>\ar[r]<4.5pt>\ar[r]<-4.5pt> &\mathbf G^{\times 2} \times \cF   \ar[r]<3pt>\ar[r]\ar[r]<-3pt>  & 
		\mathbf G \times \cF \ar[r]<1.5pt>\ar[r]<-1.5pt> & \cF,\ 
	}\]
computed in the $\infty$-category $\St \left( \Afd, \tau_{\et}, P_{\sm} \right)$. We refer to $[\cF / \mathbf G ]$ as the \emph{quotient stack objec}t of $\cF$ by $\mathbf G$. 
\end{exem}

\begin{lemma}{\cite[Section 2.3]{me1}.}
Suppose $ \mathbf G \in \St \left( \Afd, \tau_{\et}, P_{\sm} \right) $ is a smooth group object and $\cF $ is representable by a $\Q_{\ell}$-analytic space. Then the quotient stack object $[\cF/ \mathbf G]$ is representable by a geometric stack.
\end{lemma}
%
\begin{rema}
The functor $\LocSysfr$ admits a natural conjugation action of the $\Q_\ell$-analytic general linear group $ \mathbf \GLn^{\an}  \in \An_{\Q_\ell}$. 
\end{rema}

\begin{defi}
Let $\LocSys(G_X) \coloneqq [\LocSysfr / \mathbf \GLn^{\an}]$ denote the \emph{moduli stack of rank $n$ $\ell$-adic pro-\'etale local systems on $X$}. Given a continuous surjective group homomorphism $q \colon \mathrm P_X \to \Gamma$ whose
target is a finite group we define the substack of $\LocSys(G_X)$ spanned by rank $n$ $\ell$-adic pro-\'etale local systems on $X$ \emph{ramified at infinity by level $\Gamma$} as the fiber product
	\[
		\abLocSys \coloneqq \LocSys(G_X) \times_{\LocSys ( \mathrm P_X)} \LocSys(\Gamma)
	\]
\end{defi}

\begin{theorem} \label{main1}
The moduli stack $\abLocSys(G_X)$ is representable by a $\Q_{\ell}$-analytic stack.
\end{theorem}

\begin{proof}
We have a canonical map $ \abLocSys^{\mathrm{framed}}(G_X)\to \abLocSys(X)$, which exhibits the former as a smooth atlas of the latter. The result now follows formally, as explained in \cite[section 2.3]{me1}.
\end{proof}


\begin{prop}{\cite[Corollary 3.2.5]{me1}}
The functor $\LocSys(X)$ parametrizes pro-\'etale local systems of rank $n$ on $X$.
\end{prop}

\begin{proof}
The same proof of \cite[Corollary 3.2.5]{me1} applies.
\end{proof}

\section{Derived structure}
We are interested in understanding the obstruction theory of the $\Q_\ell$-analytic moduli stacks $\LocSys(G_X)$ and $\abLocSys(G_X)$. Our goal is to show that $\LocSys(G_X)$ and
$\abLocSys(G_X)$ can be naturally enhanced to \emph{derived $ \Q_{\ell}$-analytic stacks}. We compute the corresponding cotangent complexes and analyze some consequences of the existence of derived structures on theses objects.
We will use extensively the language of derived $\Q_{\ell}$-analytic geometry as developed in \cite{porta_der, porta_rep}.


\subsection{Derived enhancement of $\LocSys(X)$} Similar to \cref{section 2.3} we let $X$ be either:
\begin{enumerate}
\item A geometrically connected smooth scheme over an algebraically closed field of characteristic $p>0$ with $p \neq \ell$;
\item The spectrum of a mixed characteristic local field, K, whose residual characteristic is different from $\ell$;
\end{enumerate}



\begin{notation}
Let $Z : = (\cZ, \cO_Z) \in \dAfdl$ be a
derived $\Q_{\ell}$-affinoid space and denote 
	\[
	A_Z \coloneqq \Gamma \left(  \cO_Z^{\alg} \right) \in \CAlg_{\Q_{\ell}}
	\]
the corresponding derived ring of \emph{global sections on $Z$}, see \cite[Theorem 3.1]{porta_hom}.
\cite[Theorem 3.3.8]{me2} implies that $A_Z$ always admits a formal model, i.e., a $\ell$-complete derived
$\mathbb Z_{\ell}$-algebra $A_0 \in \CAlg_{\mathbb Z_\ell}$ such that $\left( \Spf A_0 \right)^{\mathrm{rig}} \simeq X$. Here $(-)^{\mathrm{rig}}$ denotes the rigidification functor from derived formal $\mathbb Z_\ell$-schemes to derived
$\Q_\ell$-analytic spaces, introduced in \cite[section 3]{me2}. This allow us to prove:
\end{notation}

\begin{prop}{\cite[Proposition 4.3.6]{me1}} \label{prop:enr}
The \infcat of perfect complexes on $A$, $\mathrm{Perf}(A)$, admits a natural structure of
$\cI \left( \cP \left( \cS \right) \right)$-enriched \infcat, i.e., it can be naturally upgraded to an object in the \infcat $\left(\cE \Cat \right)^{\cI}_{\cP}(\cS).$
\end{prop}


\begin{defi}
Given $\cX \in \Mon_{\mathbb E_1}^{\grp} \left( \cI \cP (\S) \right)$ we define its \emph{materialization} by the formula
	\[
		\mathrm{Mat} \left(\cX \right) := \Map_{\Mon^{\grp}_{\mathbb E_1} \left( \mathcal{C} \right) } \left( *, \cX \right) \in \Mon_{\mathbb E_1} \left( \S \right),
	\]
where $* \in \cI \cP(\S)$ denotes the terminal object.
This formula is functorial and therefore it defines a \emph{materialization functor} $\Mat \colon \Mon_{\mathbb E_1}^{\grp} \left( \cC \right) \to \Mon_{\mathbb E_1}^{\grp} \left( \S \right)$.
\end{defi}
As a consequence of \cref{prop:enr}, there exists an object $\mathbf{ \mathrm B \GLn }   (A_Z) \in \Mon_{\mathbb E_1} \left( \cI \cP (\S)\right)$, functorial in $ Z \in \dAfdl$, such that its \emph{materalization}
	\begin{equation} \label{eq:BGLn}
		\mathrm{Mat} \left(\mathbf{\mathrm B \GLn} (A_Z) \right)  \simeq \mathrm B \GLn(A_Z) \in \Mon_{\mathbb E_1} \left( \S \right).
	\end{equation}
The right hand side of \eqref{eq:BGLn} denotes the usual Bar-construction applied to $\mathbb E_1$-group like object $\GLn(A_Z)$, $\mathrm B \GLn(A_Z) \in \Mon_{\mathbb E_1} \left( \S \right)$. See \cite[\textsection 4.3 and \textsection
4.4]{me1} for more details.


\begin{defi}{\cite[Notation 3.6.1]{lurieDAGXIII}}
We shall denote $\Sh^{\et}(X)$ the \emph{\'etale shape of $X$} defined as the fundamental groupoid associated to the $\infty$-topos $\Shv_{\text{pro-\'et}} \left( X \right)^\wedge$, of hyper-complete pro-\'etale sheaves on $X$.
\end{defi}


\begin{defi}
Let $X$ be as above. We define the \emph{derived moduli stack of $\ell$-adic pro-\'etale local systems of rank $n$ on $X$} as the functor
	\[
		\dLocSys(X) \colon \mathrm{dAfd}_{\mathbb{Q}_\ell}^{\op} \to \S, 
	\]
given informally on objects by the formula 
	\[
		 Z \in  \mathrm{dAfd}_{\mathbb{Q}_\ell}^{\op} \Mapsto \lim_{n \geq 0 } \Map_{\mathrm{Mon}_{\mathbb E_1}^{\mathrm{grp}}(\mathcal{C})} \left( \mathrm{Sh}^{\et}(X),  \mathbf{ \mathrm{B}\GLn} \left(\tau_{\leq n } (A_Z)  \right) \right),
	\]
where $\tau_{\leq n}$ denotes the $n$-truncation functor on derived $\Q_\ell$-algebras.
\end{defi}

\begin{notation}
Given $Z \in \dAfdl$ we put $\dLocSys(X)(A_Z) \coloneqq \dLocSys(X)(Z)$. Given 
	\[
		\rho \in \dLocSys(X)(A_Z)
	\]
we will refer to it as a \emph{continuous representation of }$\Sh^{\et}(X)$ \emph{with coefficients in $A_Z$}.
\end{notation}

\begin{defi}
Let $\cX := \lim_m \cX_m \in \pro \left( \S \right)$. Given $n \geq 0$, we define the \emph{$n$-truncation of $\cX$} by applying the $n$-truncation functor $\tau_{\leq n} \colon \S \to \S_{\leq n }$ pointwise
	\[
		\tau_{\leq n} \left(  \cX \right) \coloneqq \lim_m \tau_{\leq n } \cX_m  \in \pro( \S_{\leq n}
	\]
\end{defi}

\begin{notation}
Let $\iota \colon \Afd \to \dAfdl$ denote the usual inclusion functor. Denote by
	\[
		\mathrm t_{\leq 0} \left( \dLocSys(X) \right) := \dLocSys(X) \circ \iota,
	\]
the restriction of $\dLocSys(X)$ to the full subcategory $\Afd$. Given
$Z \in \Afd^{\op}$, the object $\mathbf{ \mathrm B \GLn}(A_Z) \in \Mon_{\mathbb E_1 }^{\grp}( \cC) $ is $1$-truncated, therefore we obtain an equivalence of mapping spaces:
	\[
		\Map_{\Mon_{\mathbb E_1} \left( \cI \cP (\S)\right) } \left( \Sh^{\et} (X), \mathbf{ \mathrm B \GLn} (A_Z) \right) \simeq \Map_{\Mon_{\mathbb E_1} \left( \cI \cP (\S)\right) } \left( \tau_{\leq 1} \left( \Sh^{\et} (X) \right), \mathbf{ \mathrm B \GLn} (A_Z)\right).
	\]
We have moreover an equivalence of profinite spaces $\tau_{\leq 1} \Sh(X) \simeq \mathrm B G_X$, where $G_X$ is as in \cref{section 2.3}.
\end{notation}

\begin{prop} \label{123}
Let $X$ as above. We have a canonical equivalence in the \infcat $\St \left( \Afd, \tau_{\et}, P_{\sm} \right), $
	\[
		\mathrm t_{\leq 0} \left( \dLocSys(X) \right) \simeq \LocSys(G_X),
	\]
between the $0$-truncation of $\dLocSys(X)$ and the moduli stack of rank $n$ $\ell$-adic pro-\'etale local systems.
\end{prop}

\begin{proof}
The proof of \cite[Theorem 4.5.8]{me1} applies.
\end{proof}

\begin{defi} \label{tangent}
Let $Z \in \dAfdl^{\op}$ be a derived $\Q_\ell$-affinoid space. Let $\rho \in \dLocSys(X)(A_Z)$ be a continuous representation with values in $A_Z$.
The \emph{tangent complex} of $\dLocSys(X)$ at $\rho$ is defined as the fiber
	\[
		\mathbb T_{\dLocSys(X), \rho} \coloneqq \fib_{\rho} \left( p_{A_Z} \right)
	\]
where 
	\[ 
		p_{A_Z} \colon \dLocSys(X) ( A_Z \oplus^{\an} A_Z) \to \dLocSys(A_Z), 
	\]
is the morphism of stacks
induced from the canonical projection map $A_Z \oplus^{\an} A_Z \to A_Z$, where $A_Z \oplus^{\an} A_Z$ denotes the analytic square zero extension of $A_Z$ by itself.
\end{defi}

The derived stack $ \dLocSys$ is not, in general, representable as derived $\Q_\ell$-analytic stack, as this would entail the representability of its $0$-truncation. Nevertheless we can compute its tangent
complex explicitly:

\begin{lemma}{\cite[Proposition 4.4.9.]{me1}}
Let $\rho \in \dLocSys(X)(A_Z)$. We have a natural morphism
	\[
		\mathbb{T}_{ \dLocSys(X), \rho} \to C^*_{\emph{\et}}\left(X, \mathrm{Ad} \left( \rho \right) \right)[1] , 
	\]
which is an equivalence in the derived \infcat $\Mod_{A_Z}$.
\end{lemma}

\begin{proof}
The proof of \cite[Proposition 4.4.9]{me1} applies.
\end{proof}

\begin{defi}Consider the sub-site $X^{\tame}_{ \et}$ of the small \'etale site $X_{\et}$ spanned by those \'etale coverings $Y \to X$ satisfying condition (2) in \cref{}. We can form the $\infty$-topos $\Shv^{\tame}(X) \coloneqq \Shv \left( X^{\tame}_{\et} \right)$
of \emph{tamely ramified} \'etale sheaves on the Grothendieck site $X^{\tame}_{\et}$.
\end{defi}

\begin{lemma}
There exists a natural geometric morphism of $\infty$-topoi $g_* \colon \Shv_{\emph{\et}}(X) \to \Shv^{\tame}_{\emph{\et}}(X)$  is moreover fully faithful.
\end{lemma}

\begin{proof} Let $\iota_{\tame} \colon X^{\tame}_{\et} \to X_{\et}$ denote the canonical inclusion functor. The statement follows if we show that restriction along $\iota_{\tame}$ at the level of presheaves,
	\[
		\iota_{\tame}^* \colon \Psh \left( X_{\et} \right) \to \Psh \left( X^{\tame}_{\et} \right)
	\]
 admits a fully faithful right adjoint. The existence of a right adjoint for $\iota_{\tame}^*$, denoted $\iota_{\tame, *}$, follows from the fact the adjoint functor theorem. The required right adjoint is moreover computed by means of a right Kan extension along $
 \iota_{\tame}$. Given $Y \in X_{\et}^{\tame}$ seen as an object of $X_{\et}$ via $\iota_{\tame}$, the comma \infcat $X_{\et,  / Y}^{\tame}$ admits a terminal object, which corresponds to $Y$ itself. Therefore, for each
 $\cF \in \Psh \left( X_{\et}^{\tame} \right)$, the natural morphism
 	\[	
		\theta_C \colon \iota^*_{\tame} \iota_{\tame, *} \cF(C) \to \cF(C)
	\]
 is an equivalence in $\Psh(X_{\et}^{\tame})$. For this reason, the composite $\iota^*_\tame \circ \iota_{\tame, *} $ is naturally equivalent to the identity functor and the result follows.
\end{proof}

\begin{defi}
Let $\Sh^{\tame}(X) \in \pro \left( \S \right)$ denote the fundamental $\infty$-groupoid associated to the $\infty$-topos $\Shv(X_{\et}^{\tame}$, which we refer to the \emph{tame \'etale homotopy type of} $X$.
\end{defi}

\begin{rema} \label{tame_vs_et}
The fact that the geometric morphism $g_* \colon \Shv(X_{\et}^{\tame} ) \to \Shv(X_{\et})$ is fully faithful implies that the canonical morphism 
	\[
		\Sh^{\tame} (X) \to \Sh^{\et}(X)
	\]
induces an equivalence of profinite abelian groups $\pi_i \left( \Sh^{\tame} (X) \right) \simeq \pi_i \left( \Sh^{\et}(X) \right)$ for each $i>1.$
Therefore one has a fiber sequence
	\[
		\mathrm B \pi_1^w(X) \to \Sh^{\et}(X) \to \Sh^{\tame}(X),
	\]
in the \infcat $\pro(\S)$.
\end{rema}


\begin{defi}
The derived moduli stack of \emph{wild (pro)-\'etale rank $n$ $\ell$-local systems on $X$} is defined as the functor $\dLocSys^w(X) \colon d \Afd^{\op} \to \S$ given informally by the association
	\[
		Z \in \dAfdl^{\op} \Mapsto \lim_{n \geq 0} \Map_{\Mon^{\grp}_{\mathbb E_1}(\mathcal{\cI ( \cP ( \S) )})} \left( \mathrm B \pi_1^w(X), \mathbf{ \mathrm B \GLn} \left( \tau_{\leq n}(A_Z) \right) \right) \in \S.
	\]
\end{defi}

\begin{rema}
The functor $ \dLocSys^w(X)$ satisfies descent with respect to (quasi)-\'etale site $( \dAfd, \tau_{\et})$, thus we can naturally consider $ \dLocSys^w(X)$ as an object of the $\infty$-category of \emph{derived stacks} $\dSt \left( \dAfd , \tau_{\et}, \right)$.
\end{rema}

Suppose now we have a surjective continuous group homomorphism $q \colon \pi_1^w(X) \to \Gamma$, where $\Gamma$ is a finite group. Such morphism induces then a well defined morphism (up to contractible indeterminacy) 
	\[
		\mathrm B q \colon \mathrm B \pi_1^w ( X) \to \mathrm B \Gamma,
	\]
Precomposition with $\mathrm B q$ induces a morphism of derived moduli stacks $\mathrm B q^* \colon \dLocSys^w(X) \to \dLocSys(\Gamma)$. Where $\dLocSys(\Gamma) \colon d \Afd \to \S$ is the functor informally defined by the association
	\[
		A \in \dAfdl \Mapsto \lim_{n \geq 0} \Map_{\Mon^{\grp}_{\mathbb E_1}( \mathcal{C})} \left( \mathrm B \Gamma, \mathbf{ \mathrm B \GLn } \left( \tau_{\leq n} (A) \right) \right).
	\]	

\begin{rema}
As $\mathrm B \Gamma \in \S^{\fc} \subseteq \pro \left( \S^{\fc} \right)$ it follows that, for each $Z \in d \Afd^{\op}$, we have a natural equivalence of mapping spaces
	\[
		\Map_{\Mon^{\grp}_{\mathbb{E}_1}(\mathcal{\cI ( \cP(\S))} )} \left( \mathrm B \Gamma, \mathbf{ \mathrm B \GLn}(A) \right) \simeq \Map_{\Mon_{\mathbb E_1}^{\grp} (\S)} \left( \mathrm B \Gamma,  \mathrm B \GLn(A_Z) \right).
	\]
Therefore the moduli stack $\dLocSys \left(\mathrm B \Gamma \right) $ is always representable by a derived $\Q_\ell$-analytic stack which is moreover equivalent to the analytification of the usual (algebraic) \emph{mapping stack}
$\underline{\mathbf{\mathrm{Map}} }\left( \mathrm B \Gamma, \mathrm B \GLn(-) \right) $. The latter is representable by an Artin stack, see \cite[Proposition 19.2.3.3.]{lurieSAG}.
\end{rema}

\begin{defi}
Let $X$ be a smooth scheme over a field $K$. We define the (derived) moduli stack of derived (pro)-\'etale local systems on $X$ wtih \emph{bounded ramification by $\Gamma$} as the fiber product
	\[
		\abdLocSys (X) : =  \dLocSys(X) \times_{ 	 \dLocSys( \mathrm B \Gamma 	) 	}  	\dLocSys^w (X)
	\]
\end{defi}

\begin{prop}
Let $q \colon \pi_1^w(X) \to \Gamma$ be a surjective continuous group homomorphism whose target is finite. Then the $0$-truncation of $\abdLocSys(X) $ is naturally equivalent to $\abLocSys(G_X) $ and therefore it is representable by a $\Q_{\ell}$-analytic
moduli stack.
\end{prop}

\begin{proof}
It suffices to prove the statement for the corresponding moduli associated to $\Sh^{\et}(X)$, $\mathrm B \pi_1^w(X)$ and $\mathrm B \Gamma$. Each of these three cases can be dealt as in \cref{123}.
\end{proof}

Similarly to the derived moduli stack $\dLocSys(X)$ we can compute the tangent complex of $\abdLocSys(X)$ explicitly. In order to do so, we will first need some preparations:

\begin{construction} \label{const:mod}
Let $\mathcal{X} \in \pro \left( \S_{\geq 1}^{\fc} \right)$ be a \emph{$1$-connective} profinite space and fix a morphism
	\[
		c \colon * \to \cX,
	\]
in  $\pro \left( \S^{\fc} \right)$, which is canonical up to contractible indeterminacy by connectedness of $X$.
Denote by $\Perf \left( \Q_{\ell} \right)$ the \infcat of perfect $\Q_{\ell}$-modules. One can canonically enhance $\Perf(\Q_{\ell})$ to an object in the \infcat $\cE \Cat (\cI \cP(\S))$ of $\cI \cP (\S)$-enriched \infcats. Consider the full subcategory	
	\[
		\Perf_{\ell} \left( \cX \right) \coloneqq \Fun_{\cont} \left( \cX, \Perf(\Q_{\ell} ) \right)
	\]
of $\Fun \left( \Mat \left( \cX \right) , \Perf( \Q_{\ell} ) \right)$ spanned by those functors $F \colon \cX \to \Perf( \Q_{\ell})$ with $M \coloneqq F(*)$ such that the induced morphism
	\begin{equation} \label{eq:cont}
		\Omega \Mat \left(  \cX \right) \to \End \left( M \right)
	\end{equation}
is compatible with both the pro-structure on the right hand side of \eqref{eq:cont} and the ind-pro-structure on the right hand side of \eqref{eq:cont}, i.e., it is equivalent to the materialization of a morphism
	\[
		\Omega \cX \to \End \left( M \right)
	\]
in the \infcat $\Mon^{\grp}_{\mathbb E_1} \left( \cI \cP (\S) \right)$. Thanks to \cite[Corollary 4.3.23]{me1} the \infcat
$\Perf_\ell(\cX)$ is an idempotent complete stable $\Q_{\ell}$-linear \infcat which
admits a symmetric monoidal structure given by point-wise tensor product.

We can consider its \emph{ind-completion} $\Mod_{\Q_{\ell}}(\cX) \coloneqq \ind \left(\Perf_\ell(\cX)  \right)$, which is a presentable stable symmetric monoidal
$\Q_\ell$-linear \infcat, \cite[Corollary 4.3.25]{me1}. We have a canonical functor $p_{\ell} ( \cX ) \colon \Mod_{\Q_\ell} (\cX) \to \Mod_{\Q_\ell}$ given informally by the formula
	\[
		\colim_i F_\in \Mod_{\Q_\ell} \left( \cX \right) \Mapsto \colim_i \left( F_i(*)  \right) \in \Mod_{\Q_\ell},
	\]
which we refer to as the \emph{underlying module functor}.
Given $Z \in \dAfdl^{\op}$ a derived $\Q_{\ell}$-analytic space we denote
$A_Z := \Gamma \left( Z \right)$ the corresponding derived ring of global sections. Consider the extension of scalars \infcat $\Mod_{A_Z} \left( \cX \right) := \Mod_{\Q_{\ell}} \left( \cX \right) \otimes_{\Q_\ell} A_Z$, which is a presentable stable symmetric
monoidal $A_Z$-linear \infcat, \cite[Corollary 4.3.25]{me1}. We can base change $p_\ell (\cX)$ to a well defined  (up to contractible indeterminacy) functor $ p_{A_Z} \left( \cX \right)  \colon \Mod_{A_Z} \left( \cX \right) \to \Mod_{A_Z}$ given informally by the
association
	\[
		\left( \colim_i F_i \right) \otimes_{\Q_{\ell}} A_Z \in \Mod_{A_Z} \left( \cX \right) \Mapsto \colim_i \left( F_i(*) \otimes_{\Q_\ell} A_Z \right) \in \Mod_{A_Z}.
	\]
We refer to $p_A \left( \cX \right)$ as the \emph{underlying module functor with coefficients in $A_Z$}.
\end{construction}

\begin{prop} \label{tang_comp}
Let $Z \in \dAfd$ and $\rho \in \abdLocSys( X)(A_Z)$, then the natural morphism of tangent complex at $\rho$
	\[
		\mathbb T_{\abdLocSys,  \rho} \to \mathbb T_{\dLocSys,  \rho}
	\]
is an equivalence in the \infcat $\Mod_{A_Z}$. In particular, we have an equivalence of $A_Z$-modules
	\[
		\mathbb T_{\abdLocSys, \  \rho} \simeq C^*_{\emph{\et}} \left( X , \Ad  \left( \rho  \right)  \right) [1] \in \Mod_{A_Z}.
	\]
\end{prop}


\begin{proof}
Let  $\Pi \coloneqq \mathrm{B} q \colon \mathrm B \pi_1^w ( X) \to \mathrm B \Gamma$ denote the morphism of profinite homotopy types induced from a continuous surjective group homomorphism $q \colon \pi_1^w(X) \to \Gamma$ whose target is finite.
We can form a fiber sequence
	\begin{equation} \label{fib}
		\cY \to \mathrm B \pi_1^w(X) \to \mathrm B \Gamma
	\end{equation}
in the $\infty$-category $\pro \left( \S^{\mathrm{fc}}_{\geq 1} \right)_{*/}$ of pointed $1$-connective profinite spaces. Consider the \infcats $\Mod_A \left( \Sh^w( X) \right)$ and $\Mod_A \left( \mathrm B \Gamma \right)$ introduced in \cref{const:mod}. 
Let $\cC_{A, n} \left( \mathrm B \pi_1^w(X) \right)$ and $\cC_{A, n} \left( \mathrm B \Gamma \right)$ denote the full subcategories of $\Mod_A \left( \mathrm B \pi_1^w( X) \right)$ and $\Mod_A \left( \mathrm B \Gamma \right)$, whose underlying modules are
equivalent to rank
$n$
free $A$-modules. It is now follows from the definitions that we have equivalence of spaces 
	\[
		\dLocSys \left( \mathrm B \pi_1^w(X) \right) \simeq \cC_{A, n} \left( \mathrm B \pi_1^w(X) \right)^{\simeq} \text{ and } \dLocSys \left( \mathrm B \Gamma \right) \simeq \cC_{A, n } \left( \mathrm B \Gamma \right)^{\simeq}
	\]
where $(-)^{\simeq}$ denotes the underlying $\infty$-groupoid functor.
The fiber sequence displayed in \eqref{fib} induces an equivalence of $\infty$-categories
	\begin{equation} \label{eq:cats}
		\Mod_A \left( \mathrm B \Gamma \right) \simeq \Mod_A \left( \mathrm B \pi_1^w(X) \right)^\cY
	\end{equation}
where the right hand side of \eqref{eq:cats} denotes the $\infty$-category of $\cY$-equivariant
continuous representations of $\mathrm B \pi_1^w(X)$ with $A$-coefficients. Thanks to \cite[Proposition 4.4.9.]{me1} we have an equivalence of $A$-modules
	\begin{equation} \label{eq:tangSw}
		\mathbb T_{\dLocSys \left( \mathrm B \pi_1^w(X) \right),  \ \rho_{|_{\mathrm B \pi_1^w(X)}}} \simeq \Map_{\Mod_{A_Z} \left(\mathrm B \pi_1^w(X)\right) } \left( 1 ,  \rho_{|_{\mathrm B \pi_1^w(X)} }  \otimes \rho_{|_{\mathrm B \pi_1^w(X)}}^{\vee} \right) [1]
	\end{equation}
and similarly,
	\begin{equation}
		\mathbb T_{\dLocSys \left( \mathrm B \Gamma \right), \ \rho_{\Gamma}} \simeq \Map_{\Mod_{A_Z} \left(\mathrm B \Gamma \right) } \left( 1 ,  \rho_{\Gamma} \otimes \rho_{\Gamma}^{\vee} \right) [1]
	\end{equation}
By definition of $\rho$, we have an equivalence $\rho^\cY \simeq \rho$, where $(-)^\cY$ denotes (homotopy) fixed points with respect to the morphism $\cY \to \mathrm B \pi_1^w(X)$. Thus we obtain a natural equivalence of $A$-modules:
	\begin{equation} \label{fixed}
		 \Map_{\Mod_{A_Z} \left(\mathrm B \pi_1^w(X)\right) } \left( 1 ,  \rho \otimes \rho^{\vee} \right) [1] \simeq \Map_{\Mod_{A_Z} \left(\mathrm B \pi_1^w(X) \right) } \left( 1 , (  \rho_{\Gamma} \otimes \rho_{\Gamma}^{\vee} )^\cY \right) [1].
	\end{equation}
Homotopy $\cY$-fixed points are computed by $\cY$-indexed limits.
As the $\cY$-indexed limit computing the right hand side of \eqref{fixed} has identity transition morphisms we conclude that the right hand side of \eqref{fixed} is naturally equivalent to the mapping space
	\begin{equation}
		 \Map_{\Mod_A \left(\mathrm B \pi_1^w(X) \right) } \left( 1 , (  \rho \otimes \rho^{\vee} )^\cY \right) [1] \simeq \Map_{\Mod_A \left(\mathrm B \Gamma \right) } \left( 1 , \Pi_*(  \rho \otimes \rho^{\vee} ) \right) [1]
	\end{equation}
where $\Pi_* \colon \Mod_A \left(\mathrm B \pi_1^w(X)\right)  \to \Mod_A \left( \mathrm B \Gamma \right) $ denotes a right adjoint to the forgetful $\Pi^* \colon \Mod_A \left( \mathrm B \Gamma \right) \to \Mod_A \left( \mathrm B \pi_1^w(X) \right)$. 
As a consequence we have an equivalence
 	\begin{equation}
		 \Map_{\Mod_A \left(\mathrm B \pi_1^w(X)\right) } \left(1 ,  \rho \otimes \rho^{\vee}  \right) [1] \simeq \Map_{ \Mod_A \left(\mathrm B \Gamma \right) } \left( 1 , \Pi_*(  \rho \otimes \rho^{\vee} ) \right) [1]
	\end{equation}
in the \infcat $\S$. Notice that, by construction
	\begin{equation} \label{eq:gamma}
		\rho_{\Gamma} \otimes \rho_{\Gamma}^{\vee} \simeq \left( \rho \otimes \rho^{\vee} \right)_{\Gamma}
	\end{equation}
in the \infcat $\Mod_A \left( \mathrm B \Gamma \right)$ and 
	\begin{equation} \label{eq:comp}
		\Pi_* \left( \rho \otimes \rho^\vee \right)
		\simeq \left( \rho \otimes \rho^\vee \right)_{\Gamma},
	\end{equation}
as the restriction of $\rho \otimes \rho^\vee$ to $\cY$ is trivial. Thanks to \eqref{eq:tangSw} through \eqref{eq:comp}
we conclude that the canonical morphism $\LocSys \left( \mathrm B \Gamma \right) \to \LocSys \left( \mathrm B \pi_1^w(X) \right)$ induces an equivalence on tangent spaces, as desired.
\end{proof}

\begin{construction}
Consider the fiber sequence
	\[
		\mathrm B \pi_1^w(X) \to \Sh^{\et}(X) \to \Sh^{\tame}(X)
	\]
in the \infcat $\Mon_{\bE_1}^{\grp} \left( \pro(\S^{\fc})\right)$.
Such fiber sequence is classified by a morphism $\Sh^{\tame}(X) \to \mathrm B \aut \left( \pi_1^w(X) \right)$ in the \infcat $\Mon_{\bE_1}^{\grp} \left( \pro(\S^{\fc}) \right)$, where $\aut \left( \pi_1^w(X) \right)$ denotes the profinite group of exterior automorphisms of
$\pi_1^w(X)$. The $\aut $-construction is functorial. Therefore, given a finite quotient $q \colon \mathrm B \pi_1^w(X) \to \Gamma$ one has a canonical morphism $\theta \colon \Sh^{\tame}(X) \to \mathrm B \aut( \Gamma )$ in the \infcat
$\Mon_{\bE_1}^{\grp} \left( \pro( \S^{\fc}) \right)$. The morphism $\theta$ classifies a fiber sequence
	\[
		\mathrm B \Gamma \to \cX \to \Sh^{\tame}(X),
	\]
where $\cX \in \Mon_{\bE_1}^{\grp} \left(  \pro( \S^{\fc}) \right)$ is a suitable profinite space. Moreover, we have the diagram of fiber sequences
	\begin{equation} \label{ext_X}
	\begin{tikzcd}
		\mathrm B \pi_1^w(X) \ar{r} \ar{d} &  \Sh^{\et}(X) \ar{r} \ar{d} & \Sh^{\tame}(X) \ar[equal]{d} \\
		\mathrm B \Gamma \ar{r} & \cX \ar{r} & \Sh^{\tame}(X)
	\end{tikzcd}
	\end{equation}
in the \infcat $\Mon_{\mathbb E_1}^{\grp} \left( 	\pro(\S^{\fc}) \right)$. The left commutative square displayed in \eqref{ext_X} is a pushout diagram. The canonical morphism at the level of derived moduli stacks
	\[
		\abdLocSys(X) \to \dLocSys(\cX)
	\]
is an equivalence in $\dSt \left( \dAfdl, \tau_{\et} \right)$.
\end{construction}



\begin{theorem}
The (derived) moduli stack $\abdLocSys(X)$ is representable by a derived $\Q_{\ell}$-analytic stack.
\end{theorem}

\begin{proof}
Thanks to \cite[Theorem 7.1]{porta_rep} we need to check that the functor $\abdLocSys(X)$ has representable $0$-truncation, it admits a (global) cotangent complex and it is compatible with Postnikov towers. The representability of $t_0(\abdLocSys(X) ) \simeq
\abLocSys(X)$ follows from \cref{cor_rep}. The fact that $\abdLocSys(X)$ admits a global cotangent complex follows from the following:
	\begin{enumerate}
		\item $\abLocSys(X)$ admits a global tangent complex, which is a consequence of \cref{tang_comp}
		\item For each $\rho \in \abLocSys(X)$, $\mathbb T_{\abdLocSys(X), \rho}$ is a dualizable object of $\Mod_{A^{\alg}}$, where $A^{\alg}$ is the (derived) coefficient $\Q_{\ell}$-algebra for $\rho$. This assertion follows from \cite[Theorem 19.1]{milne_et}
		together with the proof of \cite[Proposition 3.1.7]{me1}.
	\end{enumerate}
Compatibility with Postnikov towers follows essentially in the same way as in \cite[Proposition 4.4.4. and Lemma 4.4.14.]{me1}
\end{proof}


\section{Comparison statements}

\subsection{Comparison with Mazur's deformation functor}
Let $K$ be a finite extension of $\Q_{\ell}$, $\O_K$ its ring of integers and $k := \O_K / \mathfrak{m}_K$ its residue field. We denote $\CAlg_{/k}^{\sm}$ the $\infty$-category of \emph{derived small $k$-algebras} augmented over $k$.

Let $G$ be a profinite group and $\rho \colon G \to \GLn(K)$ be a continuous $\ell$-adic representation of $G$. Up to conjugation, $\rho$ factors through $\GLn(\O_K ) \subseteq \GLn(K)$ and we can consider its corresponding
residual continuous $k$-representation
	\[
		\overline{\rho} \colon G \to \GLn(k).
	\]
The representation $\rho$ can the be obtained as the inverse limit of $\{ \overline{\rho}_n \colon G \to \GLn(\cO_K / \mathfrak{m}_K^{n+1}) \}_n$, where each $\overline{\rho}_n \simeq \rho \ \mathrm{mod } \ \mathfrak{m}^{n+1}$. For each $n \geq 0$,
$\overline{\rho}_n $ is a deformation of the residual representation $\overline{\rho}$ to the ring $\cO_K / \mathfrak m_K^{n+1}$.
Therefore, in order to understand continuous representations $\rho \colon G \to \GLn(K)$ one might hope to understand residual representations $\overline{\rho} \colon G \to \GLn(k)$ together with their corresponding
deformation theory. For this reason,
it is reasonable to consider the corresponding \emph{derived formal moduli problem}, see \cite[Definition 12.1.3.1]{lurieSAG}, associated to $\overline{\rho}$:
	\[
		\Def_{\overline{\rho}} \colon \CAlg^{\sm}_{/ k} \to \S,
	\]
given informally via the formula
	\begin{equation} \label{deform}
		A \in \CAlg^{\sm}_{/ k} \Mapsto \Map_{\Mon^{\grp}_{\mathbb E_1} \left( \cC \right)} 	 \left( \mathrm B G, \mathbf{ \mathrm B \GLn}(A) \right) \times_{ \Map_{\Mon^{\grp}_{\mathbb E_1}}(\cC)  \left( \mathrm B G,
		\mathbf{ \mathrm B \GLn} (k) \right) } \{ \overline{\rho} \} \in \S
		.
	\end{equation}


\begin{construction}
\cite[Proposition 4.2.6]{me1} and its proof imply that one can identify the tangent complex of $\Def_{\overline{\rho}}$ with the complex of continuous cochains on the adjoint representation $\Ad(\rho)$ of $\rho$
	\begin{equation} \label{cont_coh}
		\mathbb T_{\Def_{\overline{\rho}}} \simeq C^*_{\mathrm{cont}} \left( G, \Ad( \rho ) \right)[1]
	\end{equation}
in the $\infty$-category $\Mod_k$. 
Replacing $\mathrm B G$ in \eqref{deform} by \'etale homotopy type of $X$, $\Sh^{\et}(X)$, and $C^*_{\et}$ by $C^*_{\mathrm{cont}}$ in \eqref{cont_coh} it follows by \cite[Theorem 19.1]{milne_et} together with \cite[Theorem 6.2.5]{lurieDAGXII} that $
\Def_{\overline{\rho}}$ is \emph{pro-representable} by
a local Noetherian derived ring $A_{\overline{\rho}} \in \CAlg_{/ k}$ whose residue field is equivalent to $k$. Moreover, $A_{\overline{\rho}}$ is complete with respect to the augmentation ideal $\mathfrak{m}_{A_{\overline{\rho}}}$
(defined as the kernel of the homomorphism $\pi_0 \left(  A_{\overline{\rho}} \right) \to k$ of ordinary rings).
It follows that $A_{\overline{\rho}}$ admits a natural structure of a derived $W(k)$-algebra, where $W(k)$ denotes the ring of Witt vector of $k$.
As $\overline{\rho}$ admits deformations to $\O_K$, for e.g. $\rho$ itself, we have that $\ell \neq 0 $ in $\pi_0(A_{\overline{\rho}})$.
\end{construction}

\begin{notation}
Denote by $K^\unr \coloneqq \Frac \left( W(k) \right)$ the field of fractions of $W(k)$. It corresponds to the maximal unramified extension of $\Q_\ell$ contained in $K$.
\end{notation}

\begin{prop}
Let $\mathrm t_0 \left( \Def_{\overline{\rho}} \right)$ denote the $0$-truncation of the derived formal moduli problem $\Def_{\overline{\rho}}$, i.e., the restriction of $\Def_{\overline{\rho}}$ to the full subcategory of ordinary Artinian rings augement over $k$,
$\CAlg_{/k }^{\mathrm{sm},\heartsuit} \subseteq \CAlg_{/ k}^{\sm}$.
Then $\mathrm t_0 \left( \Def_{\overline{\rho}} \right)$ is equivalent to Mazur's deformation functor introduced in \cite[Section 1.2]{mazurDG} and $\pi_0(A_{\overline{\rho}})$
is equivalent to Mazur's universal deformation ring.
\end{prop}

\begin{proof}
Given $R \in \CAlg_{/k}^{\sm, \heartsuit} \subseteq \CAlg_{/ k }^{\sm}$ an ordinary (Artinian) local $k$-algebra,
the object $\mathbf{ \mathrm B \GLn}(R) \in \Mon_{\mathbb E_1}^{\grp} \left( \cI \cP (\S) \right)$ is \emph{$1$-truncated}. Therefore one has a natural equivalence of spaces
	\begin{equation} \label{eq:0}
		\mathrm{t_0} \left( \Def_{\overline{\rho}} \right) (R) \simeq \Map_{\Mon^{\grp}_{\mathbb E_1} \left( \cC \right) } \left( \mathrm B \pi_1^{\et}(X), \mathbf{ \mathrm B \GLn}(R) \right) \times_{\Def_{\overline{\rho}}(k)} \{ \overline{\rho} \}.
	\end{equation}
By construction, the ordinary $W(k)$-algebra $\pi_0(A_{\overline{\rho}})$ pro-represents the functor $\mathrm t_0 \left( \Def_{\overline{\rho}} \right) \colon \CAlg_{/k}^{\sm, \heartsuit} \to \S
$. As a consequence, the
mapping space on the right hand side of \eqref{eq:0} is $0$-truncated and the set of $R$-points corresponds to deformations of $\overline{\rho}$ valued in $R$. This is precisely Mazur's deformation functor, as introduced in \cite[Section 1.2]{mazurDG}, concluding
the proof.
\end{proof}

\subsection{Comparison with S. Galatius, A. Venkatesh derived deformation ring} In the case where $X $ corresponds to the spectrum of a maximal unramified extension, outside a finite set $S $ of primes, of a number field $L$ and $\rho \colon G_X \to \GLn(K)$
is a continuous representation, the corresponding derived $W(k)$-algebra was first introduced and extensively studied in \cite{galatius_dg}.

\subsection{Comparison with G. Chenevier moduli of pseudo-representations} In this section we will compare our derived moduli stack $\dLocSys(X)$ with the construction of the moduli of \emph{pseudo-representations} introduced in \cite{chenevier}.
We prove that $\dLocSys(X)$ admits an admissible analytic substack which is a disjoint union of the various $\Def_{\overline{\rho}}$.
Such disjoint union of deformation functors admits a canonical map to the moduli of pseudo-representations of introduced in \cite{chenevier}. Such morphism of derived stacks is obtained as the composite of the $0$-truncation functor 
followed by the morphism which associates to a continuous representation $\rho $ its corresponding pseudo-representation, see \cite[Definition 1.5]{chenevier}. Neverthless, the derived moduli stack $\dLocSys$
has more points in general, and we will provide a typical example in order to illustrate this phenomena.



\begin{prop} \label{moc}
Let $\overline{\rho} \colon \pi_1^{\emph{\et}}(X) \to \GLn(\overline{\mathbb F}_\ell)$ be a continuous residual $\ell$-adic representation.
To $\overline{\rho}$ we can attach a derived $\Q_\ell$-analytic space $\Def_{\overline{ \rho}}^{\rig } \in \dAn_{\Q_\ell}$ for which every closed point $\rho \colon \Sp K \to \Def_{\overline{\rho}}^\rig$ is equivalent to a continuous deformation of $\overline{\rho}$ over
$K$.
\end{prop}


\begin{proof}
Denote by $\dfSch$ the \infcat of \emph{derived formal schemes} over $W(k)$, introduced in \cite[section 2.8]{lurieSAG}.
The local Noetherian derived $W(k)$-algebra $A_{\overline{\rho}}$ is complete with respect to its maximal ideal $\mathfrak m_{A_{\overline{\rho}}}$. For this reason, we can consider its associated derived formal scheme $\Spf A_{\overline{\rho}} \in 
\dfSch_{W(k)}$.

Let $A \in \CAlg^{}_{W(k)}$ denote an admissible derived $W(k)$-algebra, see \cite[Definition 3.1.1]{me2}. We have an equivalence of mapping spaces
	\[
		\Map_{\dfSch_{W(k)}} \left( \Spf A, \Spf A_{\overline{\rho}} \right) \simeq \Map_{\CAlg_{W(k)}^{\ad}} \left( A_{\overline{\rho}}, A \right).
	\]
Notice that as $A$ is a $\ell$-complete topological almost of finite type over $W(k)$, the image of each $t \in \mathfrak m_{A_{\overline{\rho}}}$ is necessarily a topological nilpotent element of the ordinary commutative ring $\pi_0(A)$.
Let $\mathfrak m \subseteq \pi_0(A)$ denote a maximal ideal of $\pi_0(A)$ and let $\left( A\right)^{\wedge}_{\mathfrak{m}}$ denote the $\mathfrak m$-completion of $A$.
There exists a faithfully flat morphism of derived adic $W(k)$-algebra
	\[
		A \to A' \coloneqq \prod_{\mathfrak m \subseteq \pi_0(A)} 	\left( A \right)^\wedge_{\mathfrak{m}}
	\]
where the product is labeled by the set of maximal ideals of $\pi_0(A)$. By fppf descent we have an equivalence of mapping spaces
	\begin{equation} \label{ffdes}
		 \Map_{\CAlg^{\ad}_{W(k)}} \left( A_{\overline{\rho}}, A \right) \simeq \lim_{[n] \in \mathbf \Delta^{\op}} \Map_{\CAlg_{W(k)}^{\ad}} \left( A_{\overline{\rho}}, A'_{[n]} \right) 
	\end{equation}
where $A'_{[n]} \coloneqq A' \widehat{\otimes}_A \dots \widehat{\otimes}_A A'$ denotes the $n+1$-tensor fold of $A'$ with itself over $A$ computed in the \infcat of derived adic $W(k)$-algebras $\CAlg_{W(k)}^\ad$.
For a fixed $[n] \in \mathbf \Delta^{\op}$ we an equivalence of spaces
	\[
		\Map_{\CAlg^{\ad}_{W(k)} } \left( A_{\overline{\rho}}, A'_{[n]} \right) \simeq
		  \Def_{\overline{\rho}} \left( A'_{[n]} \right).
	\]
For each $[n] \in \mathbf \Delta^{\op}$ we obtain thus a natural inclusion morphism $\theta_{[n]} \colon \Map_{\CAlg^{\ad}_{W(k)}} \left( A_{\overline{\rho}}, A'_{[n]} \right) \to \dLocSys(X)(A'_{[n]})$.  The $\theta_{[n]}$ assemble together and by fppf descent induce
a morphism $\theta \colon \Map_{\CAlg^{\ad}_{W(k)}} \left( A_{\overline{\rho}}, A \right) \to \dLocSys(X) (A)$. By construction, $\theta$ induces a natural map of mapping spaces	
	\[
		\Map_{\CAlg^{\ad}_{W(k)}} \left( A_{\overline{\rho}}, A \right) \to \prod_{\mathfrak m \subseteq \pi_0(A)} \left( \dLocSys(X)(A) \times_{\Def_{\overline{\rho}} \left( A^\wedge_{\mathfrak m} \right)} \dLocSys(X)(A^\wedge_{\mathfrak m}) 
		\right)
	\]
which is equivalence of spaces.
In order words $\Spf A_{\overline{\rho}}$ represents the moduli functor which assigns to each affine derived formal scheme
$\Spf A$, over $W(k)$, the space of continuous representations $ \rho \colon \Sh^\et(X) \to \mathrm B \GLn(A)$ such that for each maximal ideal $\mathfrak m \subseteq \pi_0(A)$ the induced representation 
	\[
		\left( \rho \right)^\wedge_{\mathfrak{m}} \colon \Sh^\et(X) \to \mathrm B \GLn \left( \left( A^\wedge_{\mathfrak{m}} \right) \right)
	\]
is a deformation of $\overline{\rho} \colon \Sh^{\et}(X) \to \mathrm B \GLn(k)$. The formal spectrum $\Spf A_{\overline{\rho}}$ is locally admissible, see \cite[Definition 3.1.1]{me2}. We can thus consider its rigidificiation introduced in
\cite[Proposition 3.1.2]{me2} which we denote by $\Def^{\rig}_{\overline{\rho}}
\coloneqq \left( \Spf A_{\overline{\rho}} \right)^\rig \in \dAnl$. Notice that
$\Def^{\rig}_{\overline{\rho}}$ is not necessarily derived affinoid. 

Given $Z \in \dAfdl$, we have that any morphism $ f \colon Z \to \left( \Spf A_{\overline{\rho}} \right)^{\rig}$ admits necessarily a formal model, i.e., it is equivalent to the rigidification of a morphism
	\[
		\mathfrak f \colon \Spf A \to \Spf A_{\overline{\rho}},
	\]
where $A  \in \CAlg_{W(k)}^{\ad}$ is an admissible derived $W(k)$-algebra. The proof now follows from our previous discussion.
\end{proof}


The proof of \cref{moc} provides us with a canonical morphism of derived moduli stacks $\Def^{\rig}_{\overline{\rho}} \to \LocSys(X)$. Therefore, passing to the colimit over all continuous representations
	\[
		\overline{\rho} \colon \pi_1^{\et}(X) \to \GLn(\mathbb F_\ell)
	\]
provides us with a morphism 
	\begin{equation} \label{Psi}
		\alpha   \colon
															\coprod_{\overline{\rho} } \Def_{\overline{\rho} }^\rig
																\to
																															\LocSys(G)							
	\end{equation}
in the $\infty$-category $\dSt(\dAfdl, \tau_{\et})$.


\begin{prop} \label{open_im}
The morphism of derived $\Q_\ell$-analytic stacks $\alpha \colon \coprod_{\rho \colon G \to \GLn(\bar{\Q}_{\ell}) } \Def_{\rho}^\rig \to \LocSys(G)$ displayed in \eqref{Psi} is an open immersion of derived $\Q_{\ell}$-analytic stacks.
\end{prop}

\begin{proof}
Thanks to \cite{need reference here}, in order to prove that $\alpha$ is an open immersion it suffices to show that:
	\begin{enumerate}
		\item $\alpha$ is a monomorphism in the functor category $\mathrm{Fun} \left( \mathrm{dAfd}_{\Q_\ell}^{\op}, \S \right)$, which is an immediate consequence of the proof of \cref{moc}.
		\item The morphism $\alpha$ induces an equivalence on the corresponding cotangent complexes, or in other words the morphism $\alpha$ is \'etale. This follows immediately from our computation of both
			$\mathbb T_{\Def_{\overline{\rho}}}$ and $\mathbb T_{\LocSys(X)}$ together with \cite[Proposition 4.4.15.]{me1}.
	\end{enumerate}
\end{proof} 

\cref{open_im} implies that $\LocSys$ admits as open the disjoint union of those derived $\Q_{\ell}$-analytic spaces $\Def_{\overline{\rho}}^{\rig}$. One could then ask if $\alpha$ is itself an epimorphism of stacks and therefore an equivalence of such. However,
this is not the case in general as the following example illustrates:


\begin{exem} \label{ex_surj}
Let $G = \mathbb Z_{\ell}$ with its additive structure and let $A = \Q_{\ell} \langle T \rangle$ be the (classical) Tate $\Q_{\ell}$-algebra on one generator. Consider the following continuous representation
	\[
		\rho \colon G \to \GL_2 (\Q_{\ell} \langle T \rangle),
	\]
given by
	\[
		1 \Mapsto 
		\begin{bmatrix}
			1 & T \\
			0 & 1
		\end{bmatrix}.
	\]
It follows that $\rho$ is a $\Q_{\ell} \langle T \rangle$-point of $\LocSys( \bZ_{\ell})$ but it does not belongs to the image of the disjoint union $\Def^{\rig}_{\overline{\rho}}$ as $\rho$ cannot be factored as a point belonging to the interior of the closed unit disk
$\mathrm{Sp} \left( 	\Q_{\ell} \langle T \rangle 		\right)$.
\end{exem}

\begin{rema}
As \cref{ex_surj} suggests, $\LocSys(X)$ does admit more points than those that come from deformations of its closed points. However, we do not know if $\LocSys$ can be written as a disjoint union of the closures of $\Def^{\rig}_{\overline{\rho}}$ in
$\LocSys(X)$
\end{rema}




\section{Shifted symplectic structure on $\dLocSys$}



 
Let $X$ be a smooth and proper scheme over an algebraically closed field of positive characteristic $p>0$. Poincar\'e duality provide us with a canonical map
	\[
		\varphi \colon C^*_{\et} \left( X , \Q_\ell \right) \otimes_{\Q_\ell}  C^*_{\et} \left( X , \Q_\ell \right)  \to \Q_\ell[ 2 d]
	\]
in the derived \infcat $\Mod_{\Q_\ell}$ is non-degenerate, i.e., it induces an equivalence of \emph{derived} $\Q_\ell$-modules
	\begin{equation} \label{pd}
		 C^*_{\et} \left( X , \Q_\ell \right) \to  C^*_{\et} \left( X , \Q_\ell \right) ^\vee [2d],
	\end{equation}
in $\Mod_{\Q_\ell}$. As we have seen in the previous section, we can identify the left hand side of \eqref{pd} with a (shit) of the tangent space of $\dLocSys$ at the trivial representation. Moreover, the equivalence \label{pd} holds if we consider \'etale (co)chains
with more general coefficients, for example with $\Ad \rho$-coefficients for a continuous representation $\rho \colon \pi_1^{\et}(X) \to \GLn(A)$.

We will analyze what is entailed by \eqref{pd}, at the level of the tangent and cotangent complexes of $\dLocSys$. In order for our construction to make sense we will need to assume for the moment that an \emph{analytic HKR statements} holds in our setting.

\begin{claim}[Analytic HKR Theorem]
State the claim.
\end{claim}

This is a work in progress of the author together with F. Petit and M. Porta, which the author will provide in his PhD thesis.

\subsection{Shifted symplectic structures}

In \cite{toen_ss} the author proved the existence of shifted symplectic structures on certain derived algebraic stacks which cannot be presented as certain mapping stacks. We will thus apply the results of \cite{toen_ss} in the case of the derived
$\Q_\ell$-analytic stack $\dLocSys$. 


\begin{defi}
Consider the canonical inclusion functor $\iota \colon \dSt \left( \dAfdl, \tau_{\et}, P_{\sm} \right) \subseteq \Fun \left( \dAfdl, \S \right)$. The functor $\iota$ admits a left adjoint which we refer to as \emph{the stackification functor} $\left(- \right)^{\mathrm{st}} \colon
\Fun \left( \dAfdl, \S \right) 
\to \dSt \left( \dAfdl, \tau_{\et}, P_{\sm} \right)$.
\end{defi}

\begin{defi}
Consider the functor $\big( \PerfSys \big)^f  \colon \dAfdl \to \cS$ which is defined via the assignment
	\[
		Z \in \dAfdl^{\op} \Mapsto \Map_{\Mod_{\ind ( \pro(\S))} \left( \Cat \right)} \left( \Sh^{\et}(X), \underline{\mathbf{ \Perf}}(A_Z) \right) \in \S
	\]
where $\underline{\mathbf{\Perf}}(A_Z)
$ denotes the $\ind(\pro(\S))$-enriched \infcat of perfect $A_Z$-modules, which is equivalent to the subcategory of dualizable objects in the \infcat of Tate modules on $A_Z$, $\Mod^{\mathrm{Tate}}_{A_Z}$, see \cite{need reference 
here}. We define the moduli stack $\PerfSys \in \dSt \left( \dAfdl, \tau_{\et}, P_{\sm} \right)$ as the derived stack obtained from $\big( \PerfSys \big)^f$ by imposing the usual hyper-descent condition.
\end{defi}

\begin{rema}
This is an example of a moduli stack which cannot be presented as a usual mapping stack, instead one should think of it as an example of a \emph{continuous
mapping stack}.
\end{rema}

\begin{notation}
We will denote $\Cat^{\otimes}$ the \infcat of (small) symmetric monoidal \infcats.
\end{notation}

\begin{defi}
Let $\cC \in \Cat^\otimes$ be a symmetric monoidal \infcat. We say that $\cC$ is a rigid symmetric monoidal \infcat if every object $C \in \cC$ is dualizable.
\end{defi}

\begin{notation} \label{rigCat}
We denote by $\rigCat$ the \infcat of small rigid symmetric monoidal \infcats.
\end{notation}

 Consider the usual inclusion of \infcats $\S \hookrightarrow \Cat$, it admits a right adjoint, denoted
 	\[
		(-)^{\simeq} \colon \Cat \to \S
	\]
which we refer as the \emph{underlying $\infty$-groupoid functor}. Given $\cC \in \Cat$ its underlying $\infty$-groupoid $\cC^{\simeq} \in \S$ consists of the maximal subgroupoid of $\cC$, i.e., the subcategory spanned by equivalences in $\cC$.

\begin{lemma} \label{lem:rigidity}
There exists a valued $\Cat^{\mathrm{st}, \omega, \otimes}$-valued stack $\overKPerf \colon \dAfdl \to \Cat$ such that the composite $(-)^{\simeq} \circ \overKPerf$ is naturally equivalent to $\PerfSys$ in the \infcat $\dSt \left( \dAfdl, \tau_{\et}, P_{\sm} \right)$.
\end{lemma}

\begin{proof}
This is a consequence of construction \cite{me1}.
\end{proof}

We are thus in the situation of \cite[$\textsection$ 3]{toen_ss}.

\begin{defi}
Define $H \left( \overKPerf \right) \colon \dAfdl^{\op} \to \S$ denotes the sheaf defined by the formula
	\[
		Z \in \dAfdl^{\op} \Mapsto \Map_{\overKPerf(A_Z)} \left(\mathbf 1_{\overKPerf(A_Z)}, \mathbf 1_{\overKPerf(A_Z)} \right) \in \S,
	\]
where $\mathbf 1_{\overKPerf(A_Z)}$ denotes the unit for the corresponding symmetric monoidal structure on $\overKPerf(A_Z)$.
\end{defi}

\begin{notation}
Given $Z \in \dAfdl$ we denote simply $\mathbf 1 \coloneqq \mathbf 1_{\overKPerf(A_Z)}$ the unit of the corresponding symmetric monoidal structure on $\overKPerf(A_Z)$. Given $\rho \in \overKPerf(A_Z)$ we denote by 
	\[
		\Ad(\rho) \coloneqq \rho \otimes \rho^\vee
	\]
the \emph{internal endomorphisms object} of $\rho $ in $\overKPerf(A_Z)$.
\end{notation}

\begin{defi}
Let $\cO \colon \dAfdl^{\op} \to \CAlg_{\Q_\ell}$ given by the association
	\[
		Z \in \dAfdl^{\op} \Mapsto A_Z \coloneqq \Gamma \left( Z \right) \in \CAlg_{\Q_\ell}
	\]
\end{defi}

\begin{construction} \label{const:pair}
We can thus define a \emph{pre-orientation}, \cite[Definition 3.3]{toen_ss}, on $\overKPerf$,
	\[
		\theta \colon H \left( \overKPerf \right) \to \cO[-2d],
	\]
as follows: Let $Z \in \dAfdl$ be a derived $\Q_\ell$-affinoid space. We have a canonical equivalence in the \infcat $\Mod_{A_Z}$
	\begin{equation} \label{eq:ii}
		 \beta_{A_Z} \colon \Map_{\overKPerf(A_Z)} \left(\mathbf 1, \mathbf 1 \right) \simeq C^*_{\text{pro-\et}} \left( X, A_Z \right),
	\end{equation}
by the very construction of $\overKPerf(A_Z)$. Moreover, the projection formula produces a canonical equivalence
	\[
		C^*_{\text{pro-\et}} \left( X, A_Z \right) \simeq C^*_{\text{pro-\et}} \left( X, \Q_\ell \right) \otimes_{\Q_\ell} A_Z
	\]
in the \infcat $\Mod_{\Q_\ell}$. As $X$ is a connected smooth scheme of dimension $d$ over an algebraically closed field we have a canonical map on cohomology groups
	\[
		\Q_\ell \simeq H^0 \left( X_{\et}, \Q_\ell \right) \otimes H^{2d} \left( X_{\et}, \Q_\ell \right) \to \Q_\ell 
	\]
which induces a well defined (up to contractible indeterminacy) morphism, at the level of cochains, in the stable \infcat $\Mod_{\Q_\ell}$
	\begin{equation} \label{eq:iii}
		C^*_{\et}(X, \Q_\ell) \to \Q_\ell [-2d].
	\end{equation}
\eqref{eq:ii} together with base change of \eqref{eq:iii} along the morphism $\Q_\ell \to A_Z $ provide us with a natural morphism
	\[
	 	\Map_{\overKPerf(A_Z)} \left(\mathbf 1, \mathbf 1 \right) \to A_Z [-2d].
	\]
By naturality we obtain thus the desired pre-orientation
	\[
		\theta \colon H \left( \overKPerf \right) \to \cO[-2d].
	\]
Moreover, given $Z \in \dAfdl$ the \infcat $\overKPerf(A_Z)$ is rigid. Thus for a given object $\rho \in \overKPerf(A_Z)$ we have a canonical trace map
	\[
		\mathrm{tr}_{\rho} \colon \Ad \left( \rho \right) \to \mathbf{1}_{		} .
	\]
which together with the symmetric monoidal structure provide us with a composite of the form
	\begin{align}
		 \Map_{\overKPerf(A_Z)} \left( 	\mathbf{1}_{} , \Ad(\rho) \right) \otimes \Map_{\overKPerf(A_Z)} \left(  \mathbf{1}_{} , \Ad( \rho) \right) \to & \Map_{\overKPerf(A_Z)} \left( \mathbf{1}_{} , \Ad(\rho)  \otimes \Ad (\rho) 	
		 \right)  \\ 
		\to \Map_{\overKPerf(A_Z)} \left( \mathbf 1, \Ad(\rho) \right)	 \to & \Map_{\overKPerf(A_Z)} \left( \mathbf{1}_{} , \mathbf{1}_{} \right) \to  A_Z[-2d] &	
	\end{align}
which we can right equivalently as a morphism
	\[
		C^*_{\et} \left( X, \Ad(\rho) \right) \otimes C^*_{\et} \left( X, \Ad(\rho) \right) \to A_Z[-2d],
	\]
which by our construction coincides with the base change along $\Q_\ell \to A_Z$ of the usual \emph{pairing} induced by \emph{Poincar\'e Duality}.
\end{construction}

\begin{lemma}
The pairing of \cref{const:pair}
	\[
		\Map_{\overKPerf(A_Z)} \left( \mathbf 1, \Ad (\rho) \right) \otimes \Map_{\overKPerf(A_Z)} \left( \mathbf 1, \Ad(\rho) \right) \to A_Z [-2d]
	\]
is non-degenerate. In particular, the pre-orientation $\theta \colon H \left( \overKPerf \right) \to \cO[-2d]$ is an orientation, see \cite[Definition 3.4]{toen_ss}.
\end{lemma}

\begin{proof}
This is a consequence of the non-degeneration of the Poincar\'e duality pairing on \'etale cohomology for smooth schemes over algebraically closed fields, \cite{reference needed here!}.
\end{proof}

As a corollary of \cite[Theorem 3.7]{toen_ss} one obtains the following important result:

\begin{theorem}
The derived moduli stack $\PerfSys \in \dSt \left( \dAfdl, \tau_{\et}, P_{\sm} \right)$ admits a canonical shifted symplectic structure $\omega$. Given $Z \in \dAfdl$ and $\rho \in \PerfSys(A_Z)$, the shifted symplectic structure $\omega$
is induced by \emph{\'etale Poincar\'e duality}
	\[
		C^*_{\emph{\et}}\left(X, \Ad(\rho) \right) \otimes C^*_{\emph{\et}} \left( X, \Ad(\rho) \right) \to A_Z[-2d]
	\]
\end{theorem}

\begin{proof}
This is a direct consequence of \cite[Theorem 3.7]{toen_ss} together with \cref{const:pair}.
\end{proof}
\subsection{Applications}


\begin{thebibliography}{10}
\bibitem{me1}
Ant\'onio, Jorge. "Moduli of $ p $-adic representations of a profinite group." arXiv preprint arXiv:1709.04275 (2017).

\bibitem{me2}
Ant\'onio, Jorge. "$ p $-adic derived formal geometry and derived Raynaud localization Theorem." arXiv preprint arXiv:1805.03302 (2018).

\bibitem{Bhatt_pro}
Bhatt, Bhargav, and Peter Scholze. "The pro-\'etale topology for schemes." arXiv preprint arXiv:1309.1198 (2013).

\bibitem{bommel}
Bommel, R. van. "The Grothendieck monodromy theorem." Notes for the local Galois representation seminar in Leiden, The Netherlands, on Tuesday 28 April

\bibitem{cadoret}
Cadoret, Anna. "The fundamental theorem of Weil II for curves with ultraproduct coefficients." Preprint (available under preliminary version on https://webusers. imj-prg. fr/anna. cadoret/Travaux. html).

\bibitem{chenevier}
Chenevier, G. (2014). The p-adic analytic space of pseudocharacters of a profinite group, and pseudorepresentations over arbitrary rings. Automorphic forms and Galois representations, 1, 221-285.

\bibitem{Clark}
Clark, Pete L. "Fundamental Groups in Characteristic p". Unpublished notes.

\bibitem{gee}
Emerton, M., and Gee, T. (2015). " Scheme-theoretic images" of morphisms of stacks. arXiv preprint arXiv:1506.06146.

\bibitem{fontaine_ouyang} 
Fontaine, Jean-Marc, and Yi Ouyang. "Theory of p-adic Galois representations." preprint (2008).

\bibitem{galatius_dg}
Galatius, S., and Venkatesh, A. (2018). Derived Galois deformation rings. Advances in Mathematics, 327, 470-623.

\bibitem{Gouvea}
Gouv\^ea, Fernando Q. "Deformations of Galois representations." Arithmetic algebraic geometry (Park City, UT, 1999) 9 (1999): 233-406.


\bibitem{grothendieckSGA1}
Grothendieck, Alexandre. "Rev\^etement \'etales et groupe fondamental (SGA1)." Lecture Note in Math. 224 (1971).

\bibitem{deJong_gp}
De Jong, Aise Johan. "\'Etale fundamental groups." Lecture notes taken by Pak-Hin Lee, available at \hyperref[deJong]{https://math.columbia.edu/~phlee/CourseNotes/EtaleFundamental.pdf}.

\bibitem{deJong_etale}
De Jong, Aise Johan. "\'Etale fundamental groups of non-Archimedean analytic spaces." Compositio mathematica 97.1-2 (1995): 89-118.

\bibitem{lurieHA}
Lurie, Jacob. Higher algebra. (2012): 80.

\bibitem{lurieHTT}
Lurie, Jacob. Higher Topos Theory (AM-170). Vol. 189. Princeton University Press, 2009.

\bibitem{lurieDAGX}
Lurie, J. (2011). Formal moduli problems. Pr\'epublication accessible sur la page de l'auteur: http://www. math. harvard. edu/lurie.

\bibitem{lurieDAGXII}
Lurie, J. DAG XII: Proper morphisms, completions, and the Grothendieck existence theorem. 2011.

\bibitem{lurieDAGXIII}
Lurie, Jacob. "DAG XIII: Rational and p-adic homotopy theory. 2011."

\bibitem{lurieSAG}
Lurie, Jacob. "Spectral algebraic geometry." Preprint, available at www. math. harvard. edu/~ lurie/papers/SAG-rootfile. pdf (2016).


\bibitem{mazurDG}
Mazur, Barry. "Deforming galois representations." Galois Groups over ?. Springer, New York, NY, 1989. 385-437.

\bibitem{milne_et}
Milne, J. S. (1998). Lectures on \'etale cohomology. Available on-line at http://www. jmilne. org/math/CourseNotes/LEC. pdf.

\bibitem{nagata}
Nagata, M. (1962). Imbedding of an abstract variety in a complete variety. Journal of Mathematics of Kyoto University, 2(1), 1-10.

\bibitem{porta_hom}
Porta, M., and Yu, T. Y. (2018). Derived Hom spaces in rigid analytic geometry. arXiv preprint arXiv:1801.07730.

\bibitem{porta_der}
Porta, M., and Yu, T. Y. (2018). Derived non-archimedean analytic spaces. Selecta Mathematica, 24(2), 609-665.

\bibitem{porta_rep}
Porta, M. and  Yu, T. Y. (2017). Representability theorem in derived analytic geometry. arXiv preprint arXiv:1704.01683.

\bibitem{Pries}
Pries, Rachel J. "Wildly ramified covers with large genus." Journal of Number Theory 119.2 (2006): 194-209.

\bibitem{toen_ss}
To\"en, B. (2018). Structures symplectiques et de Poisson sur les champs en cat\'egories. arXiv preprint arXiv:1804.10444.

\bibitem{toen_ss1}
Pantev, T., To\"en, B., Vaqui\'e, M., and Vezzosi, G. (2013). Shifted symplectic structures. Publications math\'ematiques de l'IH\'ES, 117(1), 271-328.

\bibitem{stacks}
de Jong, A. J. Stacks Project. URL: http://stacks. math. columbia. edu/(visited on 04/01/2016).


\end{thebibliography}
\end{document}