\documentclass[10pt,a4paper]{amsart}
\usepackage{a4wide}
\usepackage{times}
\usepackage{amsmath,amssymb,url,stmaryrd,enumerate,bbm,enumerate}
%\usepackage[francais]
\usepackage[utf8]{inputenc}
\usepackage[all]{xy}
\usepackage{tikz-cd}
\usepackage[english]{babel}
%\usepackage[pdftex]{graphicx}
%\usepackage{graphicx}
%\usepackage{epic,eepic}
%\usepackage{epsfig}
%\usepackage{srcltx}
%\usepackage{color}

%\usepackage{times}

\usepackage{xr}
\usepackage{hyperref}
\usepackage{cleveref}

\usepackage{todonotes}

%\setcounter{secnumdepth}{1}

%% \usepackage[color]{showkeys}

%% \definecolor{refkey}{rgb}{0,0.5,0}
%% \definecolor{labelkey}{rgb}{0.6,0,0}


\usepackage{mathrsfs}

\usepackage{yhmath}

%\usepackage{fourier}
%% \usepackage[T1]{fontenc}
%% \usepackage{kpfonts}


\def\R{\mathbb{R}}
\def\fib{\mathrm{fib}}
\def\Coh{\mathrm{Coh}}
\def\LocSys{\mathrm{LocSys}_{\ell, n}}
\def\abLocSys{\mathrm{LocSys}_{\ell, n}^{ \Gamma}}
\def\K{\overline{K}}
\def\Gk{\mathrm{G}_K}
\def\Gal{\mathrm{Gal}}
\def\m{\mathfrak{m}}
\def\O{\mathcal{O}}
\def\Q{\mathbb{Q}}
\def\Z{\mathbb{Z}}
\def\F{\mathbb{F}}
\def\hZ{\hat{\mathbb{Z}}}
\def\GL{\mathrm{GL}}
\def\GLn{\mathrm{GL}_n}
\def\et{\text{\'et}}
\def\Hom{\mathrm{Hom}}
\def\cont{\mathrm{cont}}
\def\op{\mathrm{op}}
\def\Set{\mathrm{Set}}
\def\Afd{\mathrm{Afd}_{\mathbb{Q}_\ell}}
\def\tame{\mathrm{tame}}
\def\S{\mathcal{S}}
\def\fc{\mathrm{fc}}
\def\pro{\mathrm{Pro}}
\def\map{\mathrm{Map}}
\def\Sh{\mathrm{Sh}}
\def\St{\mathrm{St}}


\newcommand{\Bmu}{\mbox{$\raisebox{-0.59ex}
  {$l$}\hspace{-0.18em}\mu\hspace{-0.88em}\raisebox{-0.98ex}{\scalebox{2}
  {$\color{white}.$}}\hspace{-0.416em}\raisebox{+0.88ex}
  {$\color{white}.$}\hspace{0.46em}$}{}}

\newcommand*{\longhookrightarrow}{\ensuremath{\lhook\joinrel\relbar\joinrel\rightarrow}}

\entrymodifiers={+!!<0pt,\fontdimen22\textfont2>}

\newcommand{\overbar}[1]{\mkern 1.5mu\overline{\mkern-1mu#1\mkern-0.5mu}\mkern 1.5mu}

\newcommand{\mysetminusD}{\hbox{\tikz{\draw[line width=0.6pt,line cap=round] (3pt,0) -- (0,6pt);}}}
\newcommand{\mysetminusT}{\mysetminusD}
\newcommand{\mysetminusS}{\hbox{\tikz{\draw[line width=0.45pt,line cap=round] (2pt,0) -- (0,4pt);}}}
\newcommand{\mysetminusSS}{\hbox{\tikz{\draw[line width=0.4pt,line cap=round] (1.5pt,0) -- (0,3pt);}}}

\newcommand{\mysetminus}{\mathbin{\mathchoice{\mysetminusD}{\mysetminusT}{\mysetminusS}{\mysetminusSS}}}

\renewcommand{\setminus}{\mysetminus}

%% \DeclareMathSymbol{A}{\mathalpha}{operators}{`A}
 \DeclareMathSymbol{B}{\mathalpha}{operators}{`B}
%% \DeclareMathSymbol{C}{\mathalpha}{operators}{`C}
%% \DeclareMathSymbol{D}{\mathalpha}{operators}{`D}
%% \DeclareMathSymbol{E}{\mathalpha}{operators}{`E}
%% \DeclareMathSymbol{F}{\mathalpha}{operators}{`F}
%% \DeclareMathSymbol{G}{\mathalpha}{operators}{`G}
%% \DeclareMathSymbol{H}{\mathalpha}{operators}{`H}
%% \DeclareMathSymbol{I}{\mathalpha}{operators}{`I}
%% \DeclareMathSymbol{J}{\mathalpha}{operators}{`J}
%% \DeclareMathSymbol{K}{\mathalpha}{operators}{`K}
%% \DeclareMathSymbol{L}{\mathalpha}{operators}{`L}
%% \DeclareMathSymbol{M}{\mathalpha}{operators}{`M}
%% \DeclareMathSymbol{N}{\mathalpha}{operators}{`N}
%% \DeclareMathSymbol{O}{\mathalpha}{operators}{`O}
%% \DeclareMathSymbol{P}{\mathalpha}{operators}{`P}
%% \DeclareMathSymbol{Q}{\mathalpha}{operators}{`Q}
%% \DeclareMathSymbol{R}{\mathalpha}{operators}{`R}
%% \DeclareMathSymbol{S}{\mathalpha}{operators}{`S}
%% \DeclareMathSymbol{T}{\mathalpha}{operators}{`T}
%% \DeclareMathSymbol{U}{\mathalpha}{operators}{`U}
%% \DeclareMathSymbol{V}{\mathalpha}{operators}{`V}
%% \DeclareMathSymbol{W}{\mathalpha}{operators}{`W}
%% \DeclareMathSymbol{X}{\mathalpha}{operators}{`X}
%% \DeclareMathSymbol{Y}{\mathalpha}{operators}{`Y}
%% \DeclareMathSymbol{Z}{\mathalpha}{operators}{`Z}



\newtheorem{theorem}{Theorem}[subsection]
%\newtheorem*{theon}{Theorem}

\newtheorem{prop}{Proposition}[subsection]
%\newtheorem{prop}{Proposition}

\newtheorem{coro}{Corollary}[subsection]
%% \newtheorem*{coron}{Corollaire}

%% \newtheorem*{defin}{D?�finition}

\newtheorem{lemma}[theorem]{Lemma}
\newtheorem{question}[theorem]{Question}

\theoremstyle{definition}
\newtheorem{defi}{Definition}[section]
\newtheorem{exem}[theorem]{Example}
\newtheorem{conj}[theorem]{Conjecture}
%% \newtheorem{quest}[theo]{Question}

%\theoremstyle{remark}
\newtheorem{rema}{Remark}[section]





\author{Jorge Ant\'onio}
%\thanks{L'auteur a b�n�fici� du soutient du projet ANR-10-BLAN-0114 "ArShiFo"}
\address{Jorge Ant\'onio,  IMT Toulouse, 118 Rue de Narbonne  31400 Toulouse}
\email{jtiago1993@gmail.com}
\thanks{To all}

\begin{document}

\title{Moduli of $\ell$-adic representations (Continuation)}




\date{\today}

\maketitle

\renewcommand\labelitemi{\textbullet}






\markright{MODULI OF $\ell$-ADIC REPRESENTATIONS}


\begin{abstract}
In this text we prove that if we take $G$ is a more general profinite group, for example an absolute Galois group, $G$, the moduli $\mathrm{LocSys}_{G,n}^{ \Gamma}$ is representable by a rigid $\ell$-analytic space, provided we fix the inertia action at infinity.
\end{abstract}

\setcounter{tocdepth}{1}
\tableofcontents


\section*{Introduction}


\section{Previous works and Rappels}

\section{Setting the stage}

\subsection{Recall on the monodromy of (local) inertia}
Let $K$ be a local field, whose residual characteristic we suppose different from $\ell$. Let $\bar{K}$ be an algebraic closure of $K$ and define $\Gk := \Gal \left( \K / K \right)$ to be the absolute Galois group of $K$. Suppose we are given a finite Galois
extension $L$ of
$K$, then we can consider its Galois group $\Gal \left( L / K \right)$ together with its inertia subgroup $I_{L/K} $ which is the subgroup spanned by those elements of $\Gal \left( L /K \right)$ which fix $\O_L / \m_L$. The subgroup $I_{L/K}$ can be equivalently
defined as the kernel of the surjective group homomorphism $\Gal( L / K ) \to \Gal( l/ k)$, thus we have a short exact sequence
	\begin{equation} \label{inertia}
		1 \to I_{L/K} \to \Gal( L / K ) \to \Gal( l/ k) \to 1,
	\end{equation}
where $l $ and $k$ denote the residue fields of both $L$ and $K$, respectively. We define the (absolute) inertia group of $K$to be the inverse limit
	\[
		I_K := \lim_{L/K \text{ finite}} I_{L/ K}.
	\]
It is canonically a subgroup of $\Gk$.
Another important ingredient for us is wild inertia. Given $L/ K$ as above we can consider the subgroup $P_{L/K}$ of $I_{L/K}$ spanned by those elements which act trivially on $\O_L/ \m^2_L$ and we call it the wild inertia group (or simply wild inertia). Define then
$P_K : = \lim_{L \text{ finite}} P_{L/K}$, the absolute wild inertia group of $K$. It is a consequence of \cref{inertia} that we have a short exact sequence
	\begin{equation} \label{absinert}
		1 \to I_K \to \Gk \to G_k \to 1
	\end{equation}
where $G_k := \Gal( \bar{k}/ k)$ and $\bar{k}$ is defined as the residue field of $\K$. Notice that $P_K $ is a normal subgroup of $I_K$ and we have a short exact sequence of groups:
	\begin{equation} \label{wild_tame}
		1 \to P_K \to I_K \to I_K / P_K \to 1.
	\end{equation}
The wild inertia group $P_K$ is generally huge but it turns out that the quotient $I_K / P_K$ is much more amenable:

\begin{prop}{\cite[Corollary 13]{bommel}} \label{tame_mon}
The quotient $I_K / P_K$ is canonically isomorphic to $\hZ'(1)$, where the latter denotes the profinite group $\prod_{q \neq p} \Z_q(1)$, where $p = \mathrm{char}(k)$ is the residual characteristic.
\end{prop}

Define $P_{K, \ell} $ to be the inverse image of $\prod_{q \neq \ell, p } \Z_q$ in $I_K$. We have thus an exact sequence of groups
	\[
		1 \to P_K \to P_{K, \ell} \to \prod_{q \neq \ell, p } \Z_q \to 1.
	\]
Define $G_{K, \ell}$ to denote the quotient $G_K / P_{K, \ell}$ and notice that we have short exact sequences of groups as,
	\[
		1 \to P_{K, \ell} \to \Gk \to G_{K, \ell} \to 1,
	\]
	\[
		1 \to \Z_\ell(1) \to G_{K, \ell} \to G_k \to 1.
	\]
As a consequence, the quotient $G_{K, \ell}$ is topologically of finite type.

Suppose we are now given a continuous representation
	\[
		\rho \colon \Gk \to \GLn ( \mathbb{Q}_\ell),
	\]
(we can also consider $\rho$ with values in a finite extension $E$ of $\mathbb{Q}_\ell$, without changing the exposition). Up to conjugation we can suppose that $\rho$ actually preserves a lattice inside the vector space underlying $\rho$, thus we have a
commutative diagram
	\[
	\begin{tikzcd}
		G_K \ar{r}{\rho} \ar{dr}{} & \GLn(\Z_\ell) \ar{d} \\
						& \GLn(\Q_\ell)
	\end{tikzcd}
	\]
And $\rho \left( G_K \right) $ is a closed subgroup of $\GLn(\Z_\ell)$. We have moreover, a short exact sequence
	\[
		1 \to N_1 \to \GLn(\Z_\ell) \to \GLn(\F_\ell) \to 1,
	\]
where $N_1$ is a pro-$\ell$-subgroup of $\GLn(\Z_\ell)$, (it is the subgroup of matrices congruent to $\mathrm{Id}$ mod $\ell$). As $P_{K, \ell}$ is a profinite group which is, by construction, an inverse limit of finite groups of order prime to $\ell$ we must have,
necessarily, $\rho \left( G_K \right) \cup N_1 = \{1 \}$. Therefore $\rho( G_K)$ injects into $\GLn(\F_\ell)$ which implies that the former is a finite group, thus the wild inertia of $G_K$ acts on $\GLn(\Q_\ell)$ via a finite quotient. It is, in this way, natural to consider
the moduli of $\ell$-adic representations of $G_K$ where we fix the level by which $P_{K, \ell}$ acts on, i.e., the finite quotient of $P_{K, \ell}$ over which $\rho_{| P_{K, \ell}}$ should factor.

\subsection{Definition of the functor} Let $X$ be a smooth curve not necessarily proper. Its \'etale fundamental group $\pi_1^{\et}(X)$ is a profinite group not necessarily of topological finite generation. Let $\overline{X}$ be a smooth proper curve such that 
we have an open immersion $X \hookrightarrow \overline{X}$ and let $D = \overline{X} \backslash X = \coprod_i x_i$ be the divisor at infinity. We have an exact sequence of groups of the form
	\[
		1 \to \prod_i \Gal( \K_i / K_i) \to \pi_1^\et(X) \to \pi_1^\et \left( \overline{X} \right) \to 1, \textcolor{blue}{\mathrm{check this!}}
	\]
where $K_i$ denotes the residue field of $x_i \hookrightarrow \overline{X}$, and $\K_i$ a fixed algebraic closure. The field $K_i$ is a local field as in the previous section, so the discussion of it goes through and moreover we have that $\pi^\et_1\left(\overline(X)
\right)$ is topologically of finite generation. 

In \cite{me1} it is proven that the moduli spaces $\Hom_{\cont} \left( \pi_1^\et(\overline{X}) , \GLn \left( - \right) \right)$ and $\LocSys \left( \overline{X} \right)$ parametrizing continuous group homomorphims $\pi_1^\et (\overline{X}) \to \GLn(A)$ and pro-\'etale local
systems on $\overline{X}$ are representable by (derived) $\Q_\ell$-analytic stacks. We cannot expect the analogue functors for $X$ are representable by a geometric $\Q_\ell$-analytic space as the group $\pi_1^\et(X)$ is too big, as it contains the inertia at
infinity. Nevertheless, we can give a definition of representable functors by imposing reasonable conditions at infinity, such imposition is something natural to do after the discussion in the previous section.

For simplicity of exposition we assume that $D$ consists of only one point $x \in \overline{X}$, the general construction goes through in the same
fashion. We call $K$ the residue field of $x$ and we let $G_K := \Gal(\overline{K}_x/ K_x)$ as in the previous section. Let $\Gamma$ be a finite quotient group of $P_{K, \ell}$ and define the functor
	\[
		\Hom_\cont^{\Gamma} \left( \pi_1^\et (X), \GLn(-) \right) \colon \mathrm{Afd}_{\Q_\ell}^{\mathrm{op}} \to \mathrm{Set},
	\]
given by the assignement
	\[
		A \in \mathrm{Afd}_{\Q_\ell}^{\mathrm{op}} \mapsto \Hom_\cont^{\Gamma} \left( \pi_1^\et (X), \GLn(A) \right),
	\]
where we consider $\GLn(A)$ as a topological group making use of the natural topology on the affinoid algebra $A$ and $ \Hom_\cont^{\Gamma} \left( \pi_1^\et (X), \GLn(A) \right)$ denotes the fiber product of the diagram
	\[
	\begin{tikzcd}
		\Hom_\cont^{\Gamma} \left( \pi_1^\et (X), \GLn(A) \right) \ar{r} \ar{d} & \Hom_\cont^{} \left( \pi_1^\et (X), \GLn(A) \right) \ar{d} \\
		\Hom_\cont \left( P_{K, \ell}, \GLn(A) \right)_{\Gamma } \ar{r} & \Hom_\cont \left( P_{K, \ell}, \GLn(A) \right),
	\end{tikzcd}
	\]
where the right vertical map is the restriction along the inclusion $P_{K, \ell} \hookrightarrow G_K$ and the bottom left term denotes the set of those continuous group homomorphisms $P_{K, \ell} \to \GLn(A) $ that factor through the finite quotient $\Gamma$ when
restricted to a group homomorphism $P_{K, \ell} \to \GLn(A)$. 

\begin{theorem} \label{rep_1}
The functor $\Hom_{\cont}^\Gamma \left( \pi_1^\et(X), \GLn \right) \colon \mathrm{Afd}^{\op}_{\Q_\ell} \to \mathrm{Set}$ is representable by a rigid $\mathbb{Q}_\ell$-analytic space.
\end{theorem}

\begin{proof}
Choose a continuous group homomorphism $\varphi \colon \hZ^r \to \pi_1^\et(X)$ such that the image of the (chosen) topological generators of the former under $\varphi$ form a set of generators for $\Gamma$ seen as a quotient of $P_{K, \ell}$ and for the
(topologically finite generated) quotient $G_{K, \ell}$. Restriction under $\varphi$ induces a closed immersion of functors $\Hom_{\cont}^\Gamma \left(\pi_1^\et(X), \GLn \right) \hookrightarrow \Hom_{\cont} \left( \hZ^r, \GLn \right)$. Thanks to \cite{me1} the latter
is representable by a rigid $\Q_\ell$-analytic space and therefore also $\Hom_{\cont}^\Gamma \left( \pi_1^\et(X) , \GLn \right)$.
\end{proof}

The functor $\Hom_{\cont} \left( \pi_1^\et(X), \GLn \right) $ admits a natural action of the analytified general linear group, $\GLn^{\mathrm{an}}$, given by conjugation of morphisms $\rho \colon \pi_1^\et(X) \to \GLn(A)$. Moreover, this action preserves the condition
that a given such morphism $\rho$ factors through a finite quotient $\Gamma$, when restricted to $P_{K, \ell}$. Therefore, such action descends to the rigid $\Q_\ell$-analytic space $\Hom_{\cont}^\Gamma \left( \pi_1^\et(X), \GLn \right) $.

\begin{defi}
Define $\abLocSys \in \mathrm{St} \left( \mathrm{Afd}_{\Q_\ell}^{\op}, \tau_{\et}, P_{\mathrm{sm}} \right)$ to be the quotient stack $[\Hom_{\cont}^\Gamma \left( \pi_1^\et(X), \GLn \right)/ \GLn^{\mathrm{an}} ]$, with respect to the rigid $\Q_\ell$-analytic context, where we consider the conjugation action as above.
\end{defi}

Thanks to \cref{rep_1} we obtain the following important result:

\begin{coro} \label{rep_2}
The stack $\abLocSys$ is representable by a rigid $\Q_\ell$-analytic stack.
\end{coro}

\begin{rema}
\Cref{rep_2} is an important result as it tells us that $\abLocSys$ is an object with sufficient geometric behaviour. The reader should think of it as an analogue of an Artin stack, in the context of algebraic geometry.
\end{rema}


\subsection{Higher dimensional case} Let $X$ now be a smooth scheme over a field $k$. One can proceed similarly to the case of curves to show the representability of the moduli of $\ell$-adic local systems on $X$. In order to do so, one also needs to bound
the ramification at infinity. It turns out that method is very similar to the one used in dimension $1$, and can be used to treat homogeneously both cases. Before going through constructions we need to recall the notion of the tame fundamental group of a scheme.

It corresponds to the group of automorphisms of the functor fibre, when restricted to those tamely ramified  at infinity (finite) \'etale coverings of $X$. More precisely, let $\overline{X}$ be a smooth compactification of $X$ and define the tame fundamental group of
$X$ to be the inverse limit over all finite \'etale coverings 


Let $f \colon Y \to X$ be a finite Galois covering, with (finite) group of automorphism $\Gamma$. We define,
	\[
		\Hom_{\cont, f}^{} \left( \pi_1^\et(X), \GLn (-) \right) \colon \mathrm{Afd}_{\Q_\ell}^{\op} \to \mathrm{Set},
	\]	
as the functor which associates to each $\ell$-affinoid algebra $A$ the set $\Hom_{\cont, f}^{} \left( \pi_1^\et(X), \GLn (-) \right)$ of those continuous group homomorphisms $\rho \colon \pi_1^\et(X) \to \GLn(A)$ such that when restricted, under $f$, to $
\pi_1^\et(Y)$ factor through the tame fundamental group $\pi_1^{\tame}(Y)$. It is a fact that the latter is topologically of finite generation:

\begin{prop} \label{finite_gen_tame}
Let $Y$ be a smooth variety over a field. Then the tame fundamental group of $Y$, $\pi^\tame_1(Y)$ is topologically of finite generation.
\end{prop}

\begin{proof}
This is a formal consequence of \cite[Appendix, Theorem 1.2]{cadoret}, indeed one can find a smooth, geometrically connected curve $C$ over the base field such that the induced morphism at the level of fundamental groups
	\[
		\pi_1^{\et}(C) \to \pi_1^{\et}(X) \twoheadrightarrow \pi_1^{\mathrm{tame}}(X)
	\]
is surjective and factors through the tame quotient $\pi_1^{\et}(X) \to \pi_1^{\tame}(X)$, the latter profinite group being topologically of finite generation, \textcolor{blue}{Need reference here}. One thus concludes that also $\pi_1^{\tame}(X)$ is topological of finite
generation.
\end{proof}

Using \cref{finite_gen_tame} together with the reasoning done in the case of relative dimension $1$, we obtain the following important results:

\begin{theorem}
The functor $\Hom_{\cont, f} \left( \pi_1^\et(X), \GLn(-) \right) \colon \mathrm{Afd}_{\Q_\ell}^{\op} \to \mathrm{Set}$ is representable by a rigid $\Q_\ell$-analytic space. Moreover, its quotient stack, under the conjugation action of $\GLn^{\mathrm{an}}$ is
representable by a $\mathbb{Q}_\ell$-analytic stack.
\end{theorem}



\begin{defi}
Define $\abLocSys \in \mathrm{St} \left( \mathrm{Afd}_{\Q_\ell} , \tau_{\et}, P_{\mathrm{sm}} \right)$ as the quotient stack of $\Hom_{\cont, f} \left( \pi_1^\et(X), \GLn(-) \right)$ under the conjugation action of $\GLn^{\mathrm{an}}$ on it.
\end{defi}


\begin{coro}
The stack $\abLocSys$ is representable by a $\Q_\ell$-analytic stack.
\end{coro}

\begin{rema}
This construction works uniformly in all dimensions and it turns out that it is equivalent to our previous discussion in the case of curves.
\end{rema}


\section{Derived structure}

Let $X$ be a smooth scheme over a separably closed field of positive characteristic prime to $\ell$. To $X$ we can attach its \'etale homotopy type $\mathrm{Sh}^{\et}(X) \in \mathrm{Pro} \left( \mathcal{S}^{\mathrm{fc}} \right)$, which is a pro-object in the
$\infty$-category of finite constructible spaces, $\S^\fc$, see \cite{lurieDAGXIII} for more details. We can consider the stack in the context of derived $\Q_\ell$-analytic geometry,
	\[
		d\LocSys(X) \colon \mathrm{dAfd}_{\mathbb{Q}_\ell}^{\op} \to \S, 
	\]
given informally on objects, by the formula 
	\[
		 A \in  \mathrm{dAfd}_{\mathbb{Q}_\ell}^{\op} \mapsto \map_{\mathrm{Mon}(\mathcal{C})} \left( \mathrm{Sh}^{\et}(X), \mathrm{B}\GLn(A) \right),
	\]
where $\mathcal{C}$ is the $\infty$-category of ind-pro-objects in spaces. The space $\map_{\mathrm{Mon}(\mathcal{C})} \left( \mathrm{Sh}^{\et}(X), \mathrm{B}\GLn(A) \right)$ corresponds precisely to the $\infty$-groupoid of continuous representations of $
\mathrm{Sh}^{\et}(X)$ with values in $\mathrm{BGL}_n(A)$ equipped with its canonical \emph{ind-pro-topology}, see \cite[section 4.4]{me1} for more details about such notions. We have moreover, that 
	\[
		t_0 \left(d \LocSys(X)(A) \right) \in \St \left( d \Afd, \tau_\et, P_{\mathrm{sm}} \right)
	\]
is naturally equivalent to the stack $\LocSys(X) \colon \Afd^{\op} \to \S$ introduced previously. The derived stack $d \LocSys$ is not representable as its underlying $0$-truncation is not representable by a $\Q_\ell$-analytic stack. However, it admits a tangent
complex. Given $\rho \in d \LocSys(X)(A)$, for some $A \in d \Afd$ we have that:
	\[
		\mathbb{T}_{d \LocSys(X), \rho} \simeq C^*_{\et}\left(X, \mathrm{Ad} \left( \rho \right) \right)[1] \in \mathrm{Mod}_{A}, 
	\]
where $\mathrm{Ad} \left( \rho \right) \simeq \rho \otimes \rho^\vee$ denotes the adjoint representation and $C^*_{\et}\left(X, \mathrm{Ad} \left( \rho \right) \right) \in \mathrm{Mod}_A$ the (pro-)\'etale cohomology of $X$ with $\mathrm{Ad} (\rho)$-coefficients. It
turns out that $C^*_{\et} \left(X, \mathrm{Ad} ( \rho) \right)$ is a perfect $A$-module and therefore dualizable.














\begin{thebibliography}{10}
\bibitem{me1}
Ant\'onio, Jorge. "Moduli of $ p $-adic representations of a profinite group." arXiv preprint arXiv:1709.04275 (2017).

\bibitem{me2}
Ant\'onio, Jorge. "$ p $-adic derived formal geometry and derived Raynaud localization Theorem." arXiv preprint arXiv:1805.03302 (2018).

\bibitem{bommel}
Bommel, R. van. "The Grothendieck monodromy theorem." Notes for the local Galois representation seminar in Leiden, The Netherlands, on Tuesday 28 April

\bibitem{cadoret}
Cadoret, Anna. "The fundamental theorem of Weil II for curves with ultraproduct coefficients." Preprint (available under preliminary version on https://webusers. imj-prg. fr/anna. cadoret/Travaux. html).

\bibitem{fontaine,ouyang} 
Fontaine, Jean-Marc, and Yi Ouyang. "Theory of p-adic Galois representations." preprint (2008).

\bibitem{lurieDAGXIII}
Lurie, Jacob. "DAG XIII: Rational and p-adic homotopy theory. 2011."
\end{thebibliography}
\end{document}